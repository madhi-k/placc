%abstract german
\chapter{Zusammenfassung}

%\textit{Note: Insert the German translation of the English abstract here.}

Die Zusammenarbeit in virtuellen Teams hat die Leiter von Software-Engineering-Teams mit unerwarteten Situationen und Herausforderungen konfrontiert, die oft auf eine falsche Zuordnung von Dispositionen zu den Teammitgliedern zurückzuführen sind.  Falsche Zuschreibungen wirken sich auf die Beziehung zwischen Leiter und Mitglied aus und können zu einer alternativen Ausführung von Führungsaufgaben führen, was wiederum die Leistung des Teams beeinflusst.   In dieser Arbeit wollen wir einen Überblick über die Zuschreibungen geben, die die Projektleiter des iPraktikums im Zuge der Virtualisierung des Kurses vorgenommen haben.   Diese Forschung ist gekennzeichnet durch die Entdeckung, Kategorisierung und Modellierung von attributionsbasierten Personas nach Methoden der qualitativen und quantitativen Forschung. Letztendlich wollen wir zeigen, dass dispositionelle Attributionen weit verbreitet sind und dass sie kartiert und zu Pseudo-Charakteren zusammengefügt werden können, denen wir alle begegnet sein könnten.  Die Ergebnisse werden dann in PLACC umgesetzt, einer App, die Projektleiter mit den gesammelten Daten über Verhaltensweisen und Attributionen unterstützen soll.  Die Untersuchung eines solchen Phänomens würde nicht nur virtuelle Projektleiter mit Vorurteilen vertraut machen, von denen sie nicht wussten, dass sie sie entwickeln können, sondern könnte auch dazu führen, dass sie diese bei sich selbst erkennen und vermeiden und schließlich den grundlegenden Attributionsfehler in zukünftigen Softwareentwicklungsprojekten reduzieren.