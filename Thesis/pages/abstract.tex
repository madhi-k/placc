\chapter{\abstractname}

%The abstract summarizes your research project and serves as an overview of the following sections of your proposals. The reader should be able to instantly understand the problem and get an idea of how you are planning to solve it. Ideally, an abstract covers the following aspects and is structured accordingly:

%\begin{itemize} \item \textbf{Motivation/Objective:} Why are you going study the problem?  \item \textbf{Problem Statement:} What problem are your trying to solve? \item \textbf{Proposed Solution:} How do you want to tackle the problem? \item \textbf{Approach:} How will you conduct your research? \item \textbf{(Expected) Results:} What are the expected results of your research? \item \textbf{Conclusion:}What are your conclusions? \end{itemize}

%The motivation and objective can be completed by a preamble which introduces the problem domain and facilitates the decision whether the topic is interesting for the reader or not. A motivation should answer the following questions: why now? Materials and methods are part of the approach and describe how you accomplished your task. The result answers the ''what?'' of your written work. The conclusion summarizes your work.\\

%\noindent \textbf{Note:} Do not use citations in the abstract! \\ 

%TODO: Abstract

%Therefore, being acquainted with this phenomenon would help project leaders. Lastly, this model will support the implementation of an algorithm, that will assist project leaders with the identification of the attribution error.  


Virtual team collaboration has introduced leaders of software engineering teams to unexpected situations and challenges.  These often occur as a result of misattributing dispositions to team members.  Misattributions affect the leader-member relationship and might imply an alternate execution of leadership tasks, which consequently influence the team's performance.   In this thesis, we want to provide an overview of the attributions project leaders of the iPraktikum have created as the course turned entirely virtual.   This research is characterized by the discovery, categorization and modelling of attribution-based personas following methods of qualitative and quantitative research. Eventually, we want to demonstrate that dispositional attributions are common, and they can be mapped and attached to form pseudo-characters we might all have encountered.  The findings are then materialized in PLACC,  an app intended to support project leaders with the insights gathered on behaviours and attributions.  Research on such a phenomenon not only would introduce virtual leaders to biases previously unknown to them, but could also lead to them recognizing and preventing hasty attributions in themselves.  This would consequently reduce the judgement-based errors in future software development projects.


