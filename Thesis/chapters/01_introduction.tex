\chapter{Introduction} \label{Introduction}

Success in software engineering projects is determined by various factors, one of which is the people in a team and the environment provided to them in order to excel \cite{Reel1999}. Numerous studies point at the effect these two have in the success of software developer teams, especially when operating in an agile environment \cite{Lindsjorn2016} \cite{Chow2008}. But among the classical risks teams might face in their journey to better collaboration, cognitive biases are an often overlooked aspect that can have a negative impact on the workplace culture as well as on productivity \cite{Chattopadhyay2020}. Although the most well-known biases in software engineering teams are the ones of anchoring, optimism, and confirmation bias \cite{Mohanai2018}, in this thesis we want to investigate a cognitive bias that would relate more directly to the transformation of teams from traditional, collocated teams, into virtual teams due to the pandemic of COVID-19. More specifically, we want to investigate the role virtuality plays in the perception leaders of teams create on the team members. Such a description of cognitive bias, fits into the definition of the fundamental attribution error or attribution bias. 

This chapter will provide insights into the problematics of virtual teams and and how these are linked to the attribution bias. Simultaneously, we provide knowledge regarding this bias, in order to familiarize the reader with such a notion. Next, we will explain the solution we aim at providing to reveal the forms of attribution bias. Furthermore, we will set objectives for this thesis, which will be realized following the methodologies and approaches described in \ref{ResearchApproachAndMethodology}.

\section{Problem}

With the evolution of information and communication technologies (ICT), many organizations have established virtual teams throughout the years. Virtual teams are groups of geographically dispersed people that communicate and collaborate via different forms of technologies to accomplish organizational or project-based tasks \cite{Townsed1998}. Lately, the pandemic of 2020/21, has made digital transformation obligatory, for all businesses and for all sectors \cite{Fletcher2020}. Organizations are accelerating the adoption of digital transformation as the best way to avoid a short-term economic collapse and survive the COVID-19 pandemic with resilience \cite{Pedro2020}. Digital transformation is a complex and strategic activity that encompasses an entire organisation, and it does not solely imply the application of a series of brand new systems, such as video conferencing technologies. A major digital transformation took over the iPraktikum as well, a practical course teaching agile software development in iOS at the Technical University of Munich (TUM). All activities of this course need to be handled virtually, including a preparatory Swift programming language introductory course, meetings, important milestones, and the development process itself. The teams follow the Scrum structure, and consist of a product owner (alias project leader), scrum master (alias coach) and the scrum team of developers.

Digital transformation is accompanied by an access to new information and development of knowledge through ICT, that can transform what was once considered acceptable and unacceptable behaviours by followers, as well as by leaders \cite{Avolio2000}. Research shows, that members of short term, distributed, virtual teams are not particularly motivated to get to know their online partners \cite{Walther2002}. Moreover, without gaining such knowledge or experiencing the benefits of proper social interaction, being faced with the demands and impediments of virtual work can be challenging. In such a situation, members turn their frustration not to their own adaptation failures or to the situation, but towards their colleagues.  

In another study, Cramton \cite{Cramton2001} suggests that the dynamic underlying of such perceptions is the psychological principle called the fundamental attribution \cite{Tidwell2002}, in other words, the tendency to blame another's disposition, or personality, for what is actually a situationally-stimulated behaviour. Synonyms to this fundamental attribution error are attribution bias and correspondence bias.

The fundamental attribution has also been used to further understand the causes of leader behaviour \cite{Green1979}. As many other cognitive biases, the fundamental attribution can risk the proper management of the team, and is also correlated with a specific style of leadership, the so-called transactional leadership \cite{Masood2012}, a style not particularly successful when managing remote or virtual teams \cite{Howell2005} \cite{Purvanova2009}. So, on the one side, leaders and application of advanced information technology need to co-evolve over time to optimize a group's development and performance \cite{Avolio2000}, but on the other side, cognitive biases as result of ICT can emerge. Notably, distribution of team members has also been found to result in an increased susceptibility to the fundamental attribution error \cite{Thompson2006}. 

Considering software development teams specifically, biases are very common. Cognitive biases help to explain some of the most prominent software engineering (SE) problems in diverse activities including design, testing, requirements engineering  and project management \cite{Mohanai2018}. Topics which have received the most attention in terms of biases are information systems' usages and management, while software development, application systems require further investigation \cite{Fleischmann2014}.

As stated above, the phenomenon of fundamental attribution can cause unaware and maybe unfair ascription of team members from team leaders. The vast situations introduced by virtual collaboration in the complex domain of software engineering, and more specifically in the context of the iPraktikum, open the door to the possible rise of attributions. These can originate from stiff "You are muted" statements, or even from poor internet connection. Moreover, the fundamental attribution error is a topic, which hasn't been studied in the iPraktikum, and might play a substantial role in the relationship created between the project leader and the scrum team. Studying this phenomenon presents itself to be necessary in such a situation, and the results of this thesis and the identification of biases will help the leaders perform important tasks such as decision-making much more efficiently and impartially.

\section{Proposed Solution}

Research shows, that there is a shift in project management research from processes towards behaviours \cite{Leybourne2007}, and systematic biases are so common in the human decision-making process, that they might even cause project failure \cite{Shore2008}. Insights from  psychology, and in particular cognitive biases, can further enrich existing theories and models in information systems \cite{Fleischmann2014}, which involves software development. In this thesis, we aim at the examination of the range of behaviours a team might manifest as result of remote collaboration as perceived by the project leaders, and model the attribution process and responses. The modelling of such a process includes the identification and definition of biases, a mapping of those onto personas and an examination of the outcome these biases have on the team. The results of this research is further addressed via an algorithm, which will help future project leaders verify their biases.

The subject of this research was the iPraktikum SS21, founded by the Chair of Applied Software Engineering and this semester held by the Professorship for Digital Health at the Technical University of Munich (TUM). There were 12 teams participating, each with their own specific project and domain. The projects had 1-2 project leaders each, which played a key role in the realization of our research. Moreover, the iPraktikum fulfils one core requirement: it takes place virtually, providing the right context and environment for this thesis.


\section{Objectives}

%\textit{Note: Describe the research goals and/or research questions and how you address them by summarizing what you want to achieve in your thesis, e.g. developing a system and then evaluating it.}

To be equipped with guidance and purpose throughout the conduct of the thesis, following research questions were generated:
\begin{itemize}
	\item [RQ1] Which are the biases project leaders exhibit in the perception they have on the team members, considering the virtual situation?
	\item [RQ2] Can similarities between these perceptions be identified, and structured in the form of personas?
	\item [RQ3] How do these biases affect the way leaders react towards members of the team?
\end{itemize}

The first step in this study is to identify situations and result to an emergence of the attribution biases that are exhibited as a result of remote team collaboration. We expect biases to be present in specific teams, but also across teams. In our study we define \textit{bias} as a subjectively based tendency to prefer a given cognition over its possible alternatives \cite{Kruglanski1983}. Repetitive attributions and situations, support the creation of personas, which will include a set of various characteristics defining each persona. Ideally, there would be 4-5 main personas identified, and multiple minor ones. Additionally, we anticipate the biases to have an affect in team leader - member relationship. This can be observed during interviews, but can also be validated in post-interview procedures to fully answer RQ3.

Eventually, we want to show that different team leaders can form attributions by diverse, unique situations. Considering that most people are unaware of the biases in them, we want to introduce project leaders to a factor that might influence their decision-making. How these biases differ from one leader to the other, what can be said in general about the perceptions leaders create, and whether there is a link between this bias and leadership traits or behaviours, are the main goals of this thesis.

\section{Research Approach and Methodology} \label{ResearchApproachAndMethodology}

Considering the explorable nature the research of such a topic requires, and the numerous directions it could take, it was decided from early on that the core objective would be the identification of the attribution bias and the forms it could take. This objective is also meant to be addressed through the first research question presented in the previous section. Another characteristic of this research, is that such a phenomenon would be studied within a specific instance, the one of the iPraktikum SS21. Such circumstances, would be fitting to the following definition of a \textit{case study}: 

\begin{quote}
"Case study in software engineering is an empirical enquiry that draws on multiple sources of evidence to investigate one instance (or a small number of instances) of a contemporary software engineering phenomenon within its real-life context, especially when the boundary between phenomenon and context cannot be clearly specified." \cite{Runeson2009}
\end{quote}

The phenomenon we are aiming to study is particularly relevant as remote teams become more ubiquitous, and it will rely on real experiences, rather than simulating an experimental environment. Apart from the exploratory purpose such a research strategy should primarily fulfil, it would be interesting to gather descriptive and explanatory insights as well. In this thesis, these purposes are tackled iteratively, but a separation in three phases helps formalize and structure the processes. These will be described in Research Approach \ref{ResearchApproach}. We will also be using Grounded Theory as the overarching methodology to study data and to drive data acquisition activities within the case study, as suggested by Ferndez et al. \cite{Fernandez2004}. 

Grounded theory (GT) is a method developed by Glaser and Strauss \cite{Glaser1968}. It’s core idea is to generate theory from data, as opposed to other social research methods that are concerned with “how accurate facts can be obtained and theory tested”. The procedures of GT allow the identification of patterns in data and by analysing these patterns researchers can derive theory that is empirically valid \cite{Martin1986}.

Additionally, the reason for using the grounded theory approach is consistent with the three main reasons suggested by Benbasat et al. \cite{Benbasat1987} for using a case study strategy in IS research, namely:

\begin{enumerate}
    \item The research can study IS in a natural setting, learn the state of the art, and generate
theories from practice.
    \item The researcher can answer the questions that lead to an understanding of the nature
and complexity of the processes taking place.
    \item It is an appropriate way to research a previously little studied area.
\end{enumerate}

Therefore, this thesis will make use of mixed research methodologies to derive the required results and to abide to the framework of the case study. As already mentioned, we made use of the context of the iPraktikum SS21 to perform this study and validate it as well. Our research is supported by empirical evidence, following qualitative and quantitative research methods. Lastly, the development of an algorithmic representation of the results will give an additional quantitative approach to the realization of this thesis.

\subsection{Research Approach} \label{ResearchApproach}

The study we aim to conduct will rely on collecting data from real instances, as a case study requires. Empirical methods see the systematic collection of material or analysis of data as the way to acquire knowledge and to provide evidence to the findings. Such an approach would fit our case as well, and fits the definition of the approach of grounded theory. The process of doing a GT research study is not linear, it is rather iterative and recursive. The researcher collects, codes and analyses their initial data before further data collection/generation is undertaken \cite{Chun2019}. Furthermore, although grounded theory is mainly used for qualitative research (Glaser, 2001), it is a
general method of analysis that accepts qualitative, quantitative, and hybrid data collection from surveys, experiments, and case studies (as it is the case of this thesis) \cite{Glaser1978}.

Considering the two distinct approaches of Straussian and Glaserian grounded theory, and the objective of creating personas, we follow the Glaserian approach, in which there is a strong focus on abstract conceptualisations \cite{Fernandez2004}.

As previously mentioned, there are three phases undertaken in order to achieve the objectives of this thesis.

The first \textit{exploratory} phase, aims exactly at generating new insights and hypothesis and corresponds with the approaches of GT. It requires purposive sampling and the collection of data. Since we first want to identify biases, which are textual data representing human experience, methods of qualitative research would be the most appropriate. After the collection of data, the results must be analysed following an inductive approach. This is also referred to as constant comparative analysis, as the results (or rather codes) of each phase should be compared to each other \cite{Chun2019}. 

The second phase of the thesis relates to portraying the current status of the attribution theory, and is realized via the validation of the hypothesis derived from the previous phase. Not only do we want to gain a general overview of the situation, but through this step we aim at getting more insights regarding concrete occurrences of the attribution bias. Such a phase requires methods of quantitative research. 

The combination of both qualitative and quantitative is important in this thesis, as the topic has not been studied before in the given environment. Consequently, we are able to achieve breadth of understanding via the quantitative methods and also depth of understanding via the qualitative methods \cite{Patton2002}.

Lastly, the theories retrieved from the qualitative research activities in this thesis are then used to develop personas. These are created after the analysis of the exploratory and descriptive phases, and aim at providing more explanation regarding the phenomena. The analysis is conducted by one person (the author of this thesis). Eventually, a summary of patterns with illustrative examples, which can also be regarded as personas, is reported. These personas provide the basis for the development work that will conclude the activities of this thesis. 

\subsection{Research Methodology} \label{ResearchMethodology}

Semi-structured interviews are one of the most commonly used qualitative methods for data collection, and the method that fits our research best.

Although the interviewer prepares a list of predetermined questions, semi-structured interviews (SSI) unfold in a conversational manner offering participants the chance to explore issues
they feel are important. Semi-structured interviews are conversational and informal in tone. They allow for an open response in the participants’ own words rather than a ‘yes or no’ type answer. 

SSIs are especially relevant in the situations in which the results are not pre-defined or expected in any way. This way, during the interview in the following interesting leads can be spotted and then pursued \cite{Adams2015}. Although semi-structured, a guide must be crafted for every interview, as well as an outline of planned topics. The questions to be addressed, must be arranged in a meaningful order. The interview is a work in progress until the first trial interview, after which, the questions and agenda might be subject to modifications.

As briefly described in the previous section, data collection must be followed by data anaysis, which in such cases, is conducted via coding. 

Coding is a method used to identify concepts, similarities and conceptual occurrences in data . Coding is the pivotal link between collecting or generating data and developing a theory that explains the data \cite{Charmaz2012}. It requires for the material to be read and re-read in detail, and is an iterative process in which new codes are constantly formulated, adjusted or removed \cite{Runeson2012}. 

Interview coding is an important process, that must take into consideration the goal of the thesis, which is the exploration of participant actions/processes and perceptions found within the data and the creation of personas, as representative patterns or collections of such perceptions. Coding methods that may systematize and better reveal these, include descriptive, and/or pattern coding \cite{Wicks2017}. Referring to the literature surrounding grounded theory, one of the two basic principles is generating hypotheses using \textit{theoretical coding}. 

First step in theoretical coding is explicitly stating the research concerns and conceptual framework. The only framework used in the coding of the interview answers refers to the inappropriate behaviour categories, used during the interviews. Such code frames facilitate the process of finding common themes in accordance to the research questions of the thesis. The rest of the interviews don't contain a framework attached to them. The second step relates to identifying the phrases and sentences that provide more insights than the others. This step was conducted by highlighting and laying an emphasis on phrases that would indicate potential attribution. The third step requires a grouping of phrases or passages that revolve around similar ideas. After grouping, organizing the data into coherent categories was the next undertaken step. After coding the semi-structured interviews, the results might be summarized in a report consisting of different sections \cite{Yin2009}.

The creation of such categories, which represent early personas, would be followed by a validation phase \cite{Miaskiewicz2008}. Such a phase, which consitutes the second one in relation to the thesis, is based on quantitative methods, more specifically the one of survey questionnaire using scales. The primary goal of the this step is to verify the categories that were created previously and to identify correlations. The results of the questionnaire are analyzed via parametric statistics, which work just as well with small sample sizes, unequal variances, and non-normal distributions, without jeopardizing the conclusions \cite{Norman2010}.

Most importantly, the results will motivate the final creation of personas. Although the persona methodology has been mostly used in website design or marketing, much of the work of building personas has benefits beyond that field \cite{Madsen2014}. Personas are identified via major goals, challenges, and core details, attributes which we aim to extract and explore during the interviews. Statistical summaries are often difficult to be understood by the majority of people, hence, the employment of personas is introduced, which facilitates the illustration of the biases. Personas are an especially powerful communication tool because we as humans are naturally equipped to generate and engage with representations of people \cite{Grudin2006}. 

To concretely aid project leaders in the prevention of biases, or at least in their identification, an algorithm that maps behaviours into personas was developed. The algorithm will combine the situations leading to biases, the personas themselves, as well as the statistical quantification of the results.