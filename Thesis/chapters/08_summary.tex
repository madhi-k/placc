\chapter{Summary}

%\textit{Note: This chapter includes the status of your thesis, a conclusion and an outlook about future work.}

The topic of this thesis was conceptualized being motivated by the new virtual setting of software development projects.  It was the first time the phenomenon of dispositional attribution was being studied, and the uncertainty of coming up with interesting findings was high.  Regardless, we were able to identify the most common dispositional attributions project leaders form, as well as the behaviours triggering the most attributions.  

\section{Status}

%\textit{Note: Describe honestly the achieved goals (e.g. the well implemented and tested use cases) and the open goals here. if you only have achieved goals, you did something wrong in your analysis.}

In this section we will discuss the status of the thesis in two spectres. First we will discuss the details of the realized goals, and then the open ones, which could not be tackled within the scope of this thesis. 

We will be presenting the status on the basis on the research questions, introduced already in \ref{Objectives}: 
\begin{itemize}
	\item [RQ1] Which are the biases project leaders exhibit in the perception they have on the team members, considering the virtual situation?
	\item [RQ2] Can similarities between these perceptions be identified, and structured in the form of personas?
	\item [RQ3] How do these biases affect the way leaders react towards members of the team?
\end{itemize}

Apart from the research questions,  the development of an application that would present the findings was contemplated as one of the goals and deliverables of the thesis.  The status of its realization will also be subject to this section.

\subsection{Realized Goals}

%\textit{Note: Summarize the achieved goals by repeating the realized requirements or use cases stating how you realized them.}

The case study treated in this thesis unfolded in multiple iterations and the results were yielded after each phase of the case study.  in the introduction of this thesis we display three phases:
\begin{itemize}
	\item [1] Exploratory phase,  realized through the interviews with the project leaders
	\item [2] Descriptive phase of attributions, realized through the survey and descriptive statistics
	\item [3] Explanatory phase, characterized by the development of personas and finding answers to RQ3
\end{itemize}

A last phase was concerned with the presentation of this data in a tool, which was named \textit{ PLACC (Project Leader Attribution Companion and Consultant)}.

Through the first phase of interviews we were able to identify the behaviours that were perceived negatively from the project leaders, as well as the perception themselves.  Although attributions were the most important outcome, we were also able to provide a taxonomy of behaviours which seem to be the most problematic in the virtual environment. Although biases were identified in individual project leaders, we were able to obtain superficial information on the most common attributions across all the participants of the research, as the behaviours mentioned in the interviews would vary considerably.

In the second phase of the research we wanted to retrieve a complete,  focused and all-project-leader-inclusive dataset of attributions.  This step was important in order to generalize and analyze the before identified attributions.  Previously, they would be expressed from different individuals with experiences unalike one another.  By the means of the survey, we were able to identify the most common attributions made towards specific behaviours.  We could also identify in total, which behaviour-attribution pairs were more frequent or more likely to exist. Eventually, we could also identify the behaviours that would probably result in dispositional attribution. 

This phase served as a validation round would increase the validity of the personas too, which rely on the results of the survey.  The personas were identified in a following, third phase, by clustering together similar attributions. This similarity was defined by the type of disposition, the intention of the person,  consequences on the team to mention a few of them.  We equipped personas with a title, description,  image, attributions, behaviours and suggested reaction, as these would construct the similarities required for the development of personas.

In the same phase of the study we laid attention to the third research question as well. Directly observing or identifying reactions caused from attributions could not be attained, but we utilized the survey to make assumptions about these reactions by testing the personality trait of agreeableness.  Agreeableness is a personality trait which manifests itself in being considerate, cooperative and empathetic.  By analyzing the correlation between agreeableness and attribution tendencies we were able to form early assumptions on the team-member relationship,  and the way project leaders deal with new situations. 

In conclusion, a tool encompassing the results of the study was developed.  The interested reader can refer to~\nameref{Requirements} to recapitalize on the demands from PLACC.  The final version we were able to deliver offers the opportunity to browse through behaviours, and examine behaviour details. The behaviours offer variations, in order to cover as much ground as possible. The variations which were subject to the survey, contain attributions retrieved from the same source, accompanied by the agreeability frequencies.  These variations are also equipped with the associated personas.  Each variation also contains advice on how to deal with that specific behaviour.  Other fulfilled requirements include a tutorial for the users of PLACC and a compilation of general advice on virtual team management under the concept of a \textit{Young PL's Manual}.  

%We were able to identify behaviours that were judged negatively. We were able to create personas. We were able to look into the relationship between agreeableness and attribution tendencies, in order to explain certains reactions. 

\subsection{Open Goals}

%\textit{Note: Summarize the open goals by repeating the open requirements or use cases and explaining why you were not able to achieve them. \textbf{Important:} It might be suspicious, if you do not have open goals. This usually indicates that you did not thoroughly analyze your problems.}
During the survey analysis, we were able to derive a generic correlation between leadership traits and tendency to form attribution,  but were not able to look into attributions causing specific reactions.  A perfect result would be to associate reactions to a behaviour based on the attribution one forms as a result of the named behaviour. Normally, behaviours would need to be spontaneously exhibited in the meetings, and the project leader would have to provide their attributions and reaction right after the occurrence in order to lose as little information as possible.  We reasoned, that the realisation could not fit into the scope of the thesis, and therefore followed a more practical approach which would generalize the answer to RQ3.  We decided to rely on the personality traits of trust and agreeableness to derive such information.  Therefore,  a clear link between attributions of a certain nature and a specific reaction was not able to be found.

Another open goal relates to the application supporting this thesis, namely PLACC.  The current version of PLACC requires the researcher to manually insert behaviour and persona data in a json file, which is then decoded to be "digestible" by the application. A better approach would be providing a user interface in which researchers could add more of the research data and therefore enrich the dataset of PLACC without the domain knowledge.  Such an interface could also be used by project leaders themselves to directly share their experience. This functionality is expressed in FR5 in the~\nameref{FunctionalRequirements}.  In a more advanced version, PLACC would also store and process data from the subjects of attribution,  which in the context of iPraktikum are the team developers. In such a case, PLACC would not only take in consideration the attributions project leaders form, but also provide suggestions about the real reason behind a behaviour.  A researcher's role then, would be to maintain the model on which PLACC bases the behaviour categories, variations and personas.

Although not technically part of the research questions, the fundamental attribution error is almost always the theme of the research papers tackling dispositional attribution. This attribution could not be observed directly, as it would require a comparison of dispositional and situational attributions. ~\nameref{FutureWork} briefly discussed how this can be realized.

\section{Conclusion}

%\textit{Note: Recap shortly which problem you solved in your thesis and discuss your \textbf{contributions} here.}

The virtual setting gave rise to a multitude of unexpected conditions, including behaviours, perceptions and attributions.  In order to understand the attribution process, inappropriate behaviours were first analyzed.  After the thematic analysis of interviews, we concluded that the most disturbing behaviours are part of the following categories: \textit{participation-avoiding behaviours}, \textit{too-comfortable-at-home,} \textit{meeting stoppers} and \textit{ego-centric behaviours}. 

The interviews yielded other interesting results too. When it comes to fundamental attribution error or to over-attribute,  it was observed that the more tolerant PLs were also the ones that exhibited themselves such a behaviour.  Examples of such behaviour would be getting distracted, doing side work on the computer,  or not participating as actively.  Seemingly,  project leaders would refrain from judging a behaviour they would exhibit themselves, and, on the other side would judge more harshly behaviours, for which the PLs themselves made an effort. 

By pinpointing such a set of behaviours, we could continue with the identification of dispositional attributions.  We would notice that different variations of the same behaviour resulted in similar dispositional attributions in comparison to situational. Matching behaviours to attributions and simultaneously clustering behaviours was an iterative process that was based on the interviews. The results of this matching correspond to the survey questions and responses, which can be found in~\autoref{tab:table5} and~\autoref{tab:table6}.  Since we restricted ourselves to 14 behaviours,  we tried to generalize the behaviours in order for them to correspond to as many respondees experiences as possible.  

After retrieving the results of the questionnaire, we were able to identify the most common dispositional attributions made towards a certain behaviour.  Unanimously, the following behaviour-attribution pairs received an agreement rate of 100\%:
\begin{itemize}
	\item A team member, who is doing something else on their computers during the meeting is perceived as impolite towards the participants of the meeting.
	\item A student agreeing to too many things, often using empty phrases, but not actually delivering might be lacking the skills to keep up to their promises.
	\item A team member constantly interacting with their phone appears to be disinterested in the discussion and in what everyone else is saying.
	\item A person having a very bad sound quality renders them less approachable as a person.
\end{itemize}

We also examined the behaviours and attributions separately.  By default, we focused on behaviours that were affected by the introduction of technology in the collaboration of teams. The most attributed behaviours were Q8: Interacting with phone while in the meetings, Q14: Engaging in physical movements, Q1: Joining meetings from the phone while being on the go, Q4: Constantly turning the camera on and off,  Q3: Actively doing something besides the meeting on their computer,  Q13: Constantly interrupting or talking over another person, Q6: Agreeing to too many things whilst not delivering, Q12: Audio issues and bad sound quality. These behaviours received an over 50\% agreement rate in total.

In the work regarding personas,  each attribution was analyzed and compared to the others in order to identify commonalities in semantics, intention, social acceptance and level of intellect.  These criteria was based on the research on attribution theory, which is elaborated in~\nameref{PsychologicalBackground}.  The clusters of attributions served as the backbones of the personas. 

It was identified that most of the attributions were made in regards to unprofessionalism, impolite or inconsiderate behaviour,  indifference,  performance etc.  Eventually we were able to construct the following personas: 

\begin{itemize}
	\item The Unprofessional (5) - compilation of attributions linked to behaviours not adhering to the level of professionalism expected.
	\item Ego is the enemy (8) - attributions in this group regard the persona as ego-centric, displaying behaviours inconsiderate towards the needs and space of other team member.
	\item L'Étranger (10) - the stranger is perceived as indifferent to the importance of the course and seems detached from the team and the activities of the meeting
	\item The Loner (7) - is sometimes perceived as an outsider. Contrarily to L'Étranger, the attributions on this personas are predominately linked to a person being an introvert.
	\item The Underperformer (8) - this persona is "judged" based on their performance, commitment and experience.
	\item Hiding but not Seeking (4) - relates to attributions made when a person seems to be hiding what they are doing in the meanwhile,  and are not making visible efforts to improve.
	\item Distraction Monster (3) - getting easily distracted is what strikes the most about this persona.
\end{itemize}

The number in brackets denotes the number of attributions associated with the persona. The "cluster" with the most attributions is L'Étranger, which is an expected result considering the distance virtual teams face and the difficulty of trusting each other. More details on the personas can be found in~\nameref{Personas}.


%Dispositional attributions on people and groups are often based on salient features, such as physical characteristics (e.g., age, gender, race), noticeable behaviours (e.g., high or low performance), or markers of social categories (e.g., occupying the corner office or top floor of an office building, which often marks high status) \cite{Uleman2008}.  Considering these categories, it is obvious that the attributes are mostly made on behaviours, with instances of physical characteristics such as position in camera,  equipment, clothes and background.  The physical appearance of team members was not a prominent mention, as the project leaders did not encounter uncomfortable situations considering this domain.  Under


Attribution is a natural process which people use in different ways, to satisfy different needs.  The virtual setting presents a deficiency of social cues, and a physical distance which rendered project leader hesitant to form perceptions on the team members personality.  Nevertheless, the virtual scenario introduced new situations and new behaviours. Although few instances, the project leaders of this iPraktikum were faced with unpleasant or unconventional behaviours, unlikely to happen in face-to-face collaboration. Such situations were tested in a survey consisting of hypothetical situations., which showed a higher agreement to attributions.  22 out of 46 attributions presented in the survey received an agreement rate of at least 50\%. 

From the results we understand that attributions are common when behaviours are inconsistent with the expectations.  What is considered inappropriate is also a matter of personal flavour, and it requires deeper insights into the psychology of the project leader and their pre-defined notions of appropriate virtual setting behaviour.

The research on virtual team management and attribution theory shows that being trusting and non-judgemental seems to have benefits in the well-being of the project leader and the team too.  We believe this to be the road taken as we catch ourselves forming a dispositional attribution too. One possibility is to take the situation into account, and to acknowledge that the initial,  instinctual attributions we form might be façades for deeper issues. Although Heider described people as naive psychologists, maybe leaders should in fact be equipped with knowledge from the field of psychology. That way, leaders can assess behaviours more objectively and react accordingly. 


\section{Future Work} \label{FutureWork}

%\textit{Note: Tell us the next steps  (that you would do if you have more time. be creative, visionary and open-minded here.}

From the interviews: We take a look at behaviours and attributions in general, but future work can be focused on only one type of attribution or only one type of behaviour. This will allow more specific data to be identified, as the the time of interviews was limited, and the the size of the survey as well. 

When studying attribution, nuances can be taken into consideration.  Although we treated all attributions to have the same severity, they can be characterized with a score symbolizing the "harshness" of the statement. Such a categorization can add more clarity to the attribution tendencies.

The personas can also be studied in more detail and be compared to common anti-patterns in software engineering or project management. They can themselves serve as a basis for further anti-patterns. 

Now that inappropriate behaviours are clear there can be specific study of the fundamental attribution error. This can be realized via experimental study methods, in which project leaders can be faced with different situations,  maybe even repeated. 


PLACC is also a tool that can be extended even more.  In the scenarios section we discuss a visionary scenario in which the project leaders and developers are active actors in PLACC and provide their own perceptions continuously.  The synergy between the two parts of the coin would increase the accuracy of the results and bring even more transparency and clarify to future project leaders.  Nonetheless, we see potential in the automatisation of other processes too.  The construction of personas and the creation of the taxonomy of behaviours were manual tasks performed by the author of this thesis. Through techniques of natural language processing, a categorisation of personas and behaviours could be supported. Such techniques would clearly require considerable efforts.

Ultimately, any body of work aiming to increase awareness around dispositional attribution would be a valuable contribution to the performance and work environment of distributed software engineering teams all around the world.
