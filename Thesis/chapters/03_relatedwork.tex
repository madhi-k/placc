\chapter{Related Work}

%\textit{Note: Describe related work regarding your topic and emphasize your (scientific) contribution in \textbf{contrast} to existing approaches / concepts / workflows. Related work is usually current research by others and you defend yourself against the statement: ``Why is your thesis relevant? The problem was already solved by XYZ.'' If you have multiple related works, use subsections to separate them.}

In this thesis, the topic of attribution is studied under specific circumstances. The phenomenon is observed considering project leaders as the perceivers or creators of attributions. This study also specifically tackles software development teams. Moreover, these teams operate virtually. All these characteristic add more complexity to the task of researching these topics in conjunction with each other. Nonetheless, we were able to benefit from the research on combinations of these topics. These will be presented in the sections of this thesis.

\section{Attribution in Leadership Roles}

The fundamental attribution error in leadership is studied as early as 1979, a study in which attribution theory is presented as a vehicle for describing and understanding the causes of leader behaviour in leader-member interactions \cite{Green1979}. According to the same paper, subordinate behaviour generates leader attributions, which then stimulate certain leader behaviour. 

Which these behaviours specifically are, have been then studied by Martinko et al. 1987, identifying withdrawal of rewards, punishment or no action as possible outcome. Another stream of research examines the attribution members have on the leader of a team \cite{Sweet2020}. Furthermore, literature \cite{Gardner2018} suggests that the way individuals make attributions of others' behaviours determines how these behaviours affect both individuals' internal psychological states and external relationships with others.


\section{Attribution in virtual setting}

% Not includind software engineering teams

Attribution in virtual teams is studied within teams in various studies, but mostly in identifying it as a phenomenon. E.g. Cramton, 2002 \cite{Crampton2001} considers how the use of technology mediated communication can contribute to a less shared reality, and then to attribution. 

Atkinson et al., study the dynamics in short term distributed teams. They not only suggest that the hypothesized dynamics of outgroup attribution in distributed virtual teams does occur, but also argue that a face-to-face meeting may be valuable for people who work together online. Pauleen, 2005 \cite{Pauleen2005} argues that time lags due to technical infrastructure and technological breakdowns, if not understood by the people involved, can cause the team leader or team member to attribute non-communication to lack of manners or conscientiousness, which can then seriously affect relationships. Another study though argues, that teams recognize that constraints
may create real impediments for a their performance, and realizing their negative
impact can motivate teams to adjust, in an effort to adapt to those constraints \cite{Bazarova2012}. Bazarova et al., 2009 categorize attributions into dispositional, situational, generic situation, distance, other members or computer use \cite{Bazarova2009}, a categorization which might help frame the attributions identified in this thesis as well.


\section{Attribution in IT}

Studies on attributions in information systems consider system users as actors, in the case of user-developer misunderstandings \cite{Snead2014}, vendors and buyers of outsourcing services\cite{Rouse2007}, or attributions between information systems designers \cite{Peterson2002}. The results of these studies are similar, showing that negative outcomes such as project or negotiation failure increase the chance of negative attributions towards a counterparty or colleagues. 

Considering software engineering teams, cognitive biases help to explain many common software engineering problems in diverse activities. Such biases include estimation bias \cite{Jorgensen2012}, optimistic/pessimistic bias \cite{Fleischmann2014}, overconfidence among others, observing processes that include design, testing, requirements engineering, and project management. Research on cognitive biases has been useful not only to identify common errors, their causes and how to avoid them in SE processes, but also for developing better practices, methods, and artifacts. 

Specific work in leader attribution in remote virtual teams has been mostly conducted under the hood of relationship building, communication and trust building. Lack of situational knowledge of team members can cause misunderstandings and attribution error, leading to potential obstacles  \cite{Pauleen2005}, but from our research, attribution as a phenomenon or as a bias has not been studied separately.

