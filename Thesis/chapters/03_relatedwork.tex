\chapter{Related Work}

%\textit{Note: Describe related work regarding your topic and emphasize your (scientific) contribution in \textbf{contrast} to existing approaches / concepts / workflows. Related work is usually current research by others and you defend yourself against the statement: ``Why is your thesis relevant? The problem was already solved by XYZ.'' If you have multiple related works, use subsections to separate them.}

In this thesis, the topic of attribution is studied under specific circumstances. The phenomenon is observed considering project leaders as the perceivers or creators of attributions. This study also specifically tackles software development teams. Moreover, these teams operate and conduct all activities of software development virtually. All these characteristic add more complexity to the research of these topics in conjunction with each other. Hence, we rely on the studies made on combinations of two concepts to build a proper picture of the literature related to this thesis.  Specifically, we will discuss~\nameref{AttLeadership}, ~\nameref{AtrVirtual} and~\nameref{AttIT} separately.

\section{Attribution in Leadership Roles}\label{AttLeadership}

In \ref{VirtualLeadership}, we saw how management of virtual teams is one of the biggest challenges teams face.  The inability of virtual team members, and especially managers to observe each others actual effort tends to lead to a greater reliance on perceptions and assumptions that could be both biased and error-prone \cite{Penarroja2013}.

The fundamental attribution error in leadership is studied as early as 1979, a study in which attribution theory is presented as a vehicle for describing and understanding the causes of leader behaviour in leader-member interactions \cite{Green1979}. According to the same paper, subordinate behaviour generates leader attributions, which then stimulate certain leader behaviour.  

A literature review by Martinko et al.  (1987) presents the most prominent hypothesis by researchers of attribution in leadership - member relationship.  Some of the most interesting revelations include:

\begin{enumerate}
\item P1:  Subordinates will have a tendency to favor
external attributions for failure, whereas leaders will tend to blame internal subordinate
characteristics \cite{Mitchell1980}.
\item P2: Leaders who attribute poor member performance to effort as opposed to ability exhibit more punitive responses \cite{Mitchell1980}.
\item P3: The degree of congruence between member behaviours and leader behavioural preferences is related positively to the amount of similarity among leader and member attributions. This is a result of the tendency to attribute behaviour that they view as unusual to stable dispositional characteristics of the actor, whereas behaviours that are consistent with their own response tendencies are attributed to environmental causes \cite{Ross1977}.
\item P4: The hedonic relevance of member behaviour to a leader is related positively to the strength of the attributional biases of leaders. Since leaders are ultimately dependent upon their followers for the successful accomplishment of their objectives, the concept of hedonic relevance is especially pertinent to leader/member relations. Thus, leaders are more likely to make negative dispositional attributions about members whose behaviours they perceive as hindering as opposed to facilitating goal attainment. In effect, hedonic relevance amplifies the impact of
the biases discussed above.
\end{enumerate}

%Attribution in leadership has been studied following Kelley's and Weiner's model, but an interesting perspective is presented also by Anderson (1991) coded participants’ attributions following failure and success along thirteen dimensions. Interestingly,  interpersonalness, defined as the extent to which the cause of the event reflected on the attributer’s relationships with other people, emerged as the strongest dimension \cite{Anderson1991}.

Another stream of research examines the attribution members have on the leader of a team \cite{Sweet2020}.  Bligh et al.   found that followers’ negative views of their work environment were overly attributed to their leaders’ in that they viewed the leader as more responsible for these negative outcomes and situations than was warranted \cite{Bligh2007}.  Similarly,  Weber et al.  reported
that group success and failure were overly attributed to the leader \cite{Weber2001}.

Furthermore, literature \cite{Gardner2018} suggests that the way individuals make attributions of others' behaviours determines how these behaviours affect both individuals' internal psychological states and external relationships with others.  A separate family of theories has even emerged for the Leader-Member-Exchange and perceptions formed on these exhanges \cite{Cogliser2009}.

The scope of application for attribution theory has expanded considerably in recent years in other directions as well. Attribution processes have also emerged as an important moderator between supervisory behaviour and subordinates' reports of \textit{abusive supervision} \cite{Mackey2017}, as a factor for explaining employee entitlement \cite{Harvey2013}, and as a process underlying ethical decisions in the workplace \cite{Harvey2017}.  Attribution processes are also considered a theme to study the effects of high‐performance work systems \cite{Mackey2016}. 

Specifically, there was evidence for the occurrence of the self-serving bias when supervisors made attributions for the performance of their in-group subordinates and when both in- and out-group subordinates made attributions about their own performance. The results provided some support for the presence of the actor-observer bias when supervisors made attributions about the performance of their out-group subordinates. Consistent with prior research, these results provide yet further evidence for the positive outcomes associated with in-group status, in that in-group members are being credited with their effective performance and not blamed for their ineffective performance \cite{Campbell2006}.


\section{Attribution in virtual setting}\label{AtrVirtual}

% Not includind software engineering teams

In addition to the challenges facing traditional groups, virtual groups must adjust to temporal delays in information exchange, maintain shared context and workflow, and confront other difficulties in order to work and relate effectively.  Furthermore, the inability of virtual team members to observe each other's actual effort tends to lead to a greater reliance on perceptions and assumptions that could be both biased and erroneously negative \cite{Penarroja2013}.

Attribution in virtual teams is studied within teams in various studies.  Cramton,  2002 \cite{Cramton2002} considers how the use of technology mediated communication can contribute to a less shared reality, and then to attribution.  The failure to adapt may result in negative interpersonal judgements among group members rather than an appreciation of the socio-technical challenges of distributed work \cite{Cramton2001} .

Walther et al.  \cite{Walther2002}, study the dynamics in short term distributed teams. They not only suggest that the hypothesized dynamics of outgroup attribution in distributed virtual teams does occur, but also argue that a face-to-face meeting may be valuable for people who work together online.  Pauleen \cite{Pauleen2005} argues that time lags due to technical infrastructure and technological breakdowns, if not understood by the people involved, can cause the team leader or team member to attribute non-communication to lack of manners or conscientiousness, which can then seriously affect relationships.  Another study though argues, that teams recognize that constraints may create real impediments for a their performance, and realizing their negative impact can motivate teams to adjust, in an effort to adapt to those constraints \cite{Bazarova2012}. Bazarova et al.  categorize attributions into dispositional, situational, generic situation, distance, other members or computer use \cite{Bazarova2009}, a categorization which might help frame the attributions identified in this thesis as well.

The fundamental attribution error has been studied in virtual teams yielding mixed results.  More specifically, research in computer-mediated communication has looked at how different technologies influence the way individuals perceive their collaboration partners \cite{Straus1994}. Hancock et al. \cite{Hancock2001}, for example, found that people’s impressions of their partner's personality were more intense and less detailed in CMC than face to face. 

Lack of common location did not inflate dispositional attributions in virtual groups.  This has been specifically studied in \cite{Bazarova2009}. On the contrary, collocated members made greater dispositional judgments about partners, compared to
those of distributed groups.  The same statement holds true,  regardless of the degree of behaviour differences within their groups. These inconsistencies are not altogether surprising because the virtual group setting stretches attribution theory beyond its typical boundaries in several ways: The iterative and cumulative effects of communication with targets, the mediated nature of observations, and, particularly, the active participation rather than passive observation of the observer with the target are uncommon in traditional attribution research.

Attribution is also studied in terms of self-attribution. Interesting studies suggest that members of completely distributed groups attribute the cause of their own negative behaviour to the influence of their partners, more than do members of collocated groups. \cite{Walther2007}. 

The findings about greater dispositional attributions for collocated than distributed groups challenge the application of attribution to virtual groups that has dominated the literature \cite{Cramton2001}. There can be several explanations for the different conclusions. Several are methodological, as
detailed in their work  \cite{Cramton2001}. Other differences are more conceptual.  The
previous research on attributions in virtual groups relied on traditional actor–observer principles (the actor-observer bias is a term in social psychology that refers to a tendency to attribute one's own actions to external causes while attributing other people's behaviours to internal causes) to make predictions, whereas general attribution research has challenged the utility of these principles \cite{Bazarova2009}.

Another interesting experiment is presented by \cite{Trainer2018}, in which a software providing various information on a person's situation or disposition was employed to then investigate the attributions formed.  In conclusion,  the authors provide initial evidence that greater awareness of collaborator's responsiveness and availability information (information which is unlikely to have in a globally distributed setting \cite{Cramton2001}), can support team members in accurately explaining negative, unexpected behaviour, and calibrating their trustworthiness toward them appropriately.  The study shows that for the most part, these findings hold for both students and professionals.

Jiang et al. \cite{Jiang2011} deliver insights regarding the dispositional attribution of members showcasing self-disclosure.  This result also support the prediction that CMC participants would make more intensified interpersonal attributions when encountering high self-disclosure relative to their face-to-face counterparts. These kind of studies look specifically at behaviours that might influence attributional processes in a person. 

Specific work in leader attribution in remote virtual teams has been conducted under the hood of relationship building,  communication and trust building.  For instance, time lags due to technical infrastructure and technological breakdowns, if not understood by the people involved, can cause the team leader or team member to attribute non-communication to lack of manners or conscientiousness, which can then seriously affect relationships \cite{Cramton2002}.  Lack of situational knowledge of team members can cause misunderstandings and attribution error, leading to potential obstacles and \cite{Pauleen2005} suggests leaders to ask \textit{What are the boundary crossing influences of this situation?} before making any attribution.  Attribution theory has the potential to make significant contributions to information systems research \cite{Standing2016}.  The more we understand the attribution other form, the sooner managers can expect or prevent certain events from happening \cite{Standing2016}.

\section{Attribution in Information Technology} \label{AttIT}

Considering software engineering teams, cognitive biases help to explain common software engineering problems in diverse activities. Such biases include estimation bias, optimistic/pessimistic bias, overconfidence among others, observing processes that include design, testing, requirements engineering, and project management \cite{Jorgensen2012, Fleischmann2014}. Research on cognitive biases has been useful not only to identify common errors, their causes and how to avoid them in SE processes, but also for developing better practices, methods, and artefacts. 

Studies on attributions in information systems consider system users as actors, in the case of user-developer misunderstandings \cite{Snead2014}, vendors and buyers of outsourcing services \cite{Rouse2007}, or attributions between information systems designers \cite{Peterson2002}. The results of these studies are similar, showing that negative outcomes such as project or negotiation failure increase the chance of negative attributions towards a counterpart or colleagues. 

Understanding the attributional styles exhibited by individuals in the IT project domain can make a significant contribution to our knowledge of project management, given the limited research into individual emotions and behaviour within this domain \cite{Standing2006}. In particular, attributional style provides the opportunity to identify the important causal dimensions that affect individual emotions that lead to a behaviour consistent with mastery of the IT project domain (e.g. ability to effectively apply knowledge, skills, tools and techniques to IT projects) \cite{pmi1996}. For instance, IT support workers may attribute failure to external causes as a means of protecting their self-worth at the expense of being a potential barrier to learning \cite{Duval2002}.

Attribution in IT has been often studied under the veil of success and failure of IT projects.  Level of seniority has also been subject to research in relation to project success and failure.  According to \cite{Standing2016} employees at different levels attribute differently.  For example junior staff or support working tend to attribute success more internally than externally whilst more senior staff recognize the contribution of environmental factors more.  It could be assumed that the more experienced a person, the more balanced their attribution. 

%Line and executive managers have the tendency to increasingly make more pessimistic attributions than support workers. Support workers have the tendency to be more optimistic than line and executive managers and are more likely to overestimate their role in success but not accept responsibility for failure. \cite{Standing2006} 

%Professionals in the IT field do not attribute success and failure the same way. In particular, IT support workers and executive IT managers react differently. IT support workers attributed success more to themselves than other workers but did not attribute failure to themselves. Executives on the other hand, attribute a significant amount of failure to themselves but success to external factors. 

Attribution of success and failure to the individual or environment can be problematic when it is not balanced. \cite{Standing2012}.  So, over attribution of failure to the self, for example, can lead to pessimism and depression \cite{Standing2012}.  There is a danger in becoming too pessimistic in attributing success to external factors and that this can have a damaging effect on morale \cite{Standing2016}.  The same paper highlights the need for increasing attribution theory awareness in the leadership community.

Although the research on attribution in IT revolves often around the success rate, we assume that in a virtual scenario many of the situations and discoveries discussed in~\autoref{AtrVirtual} will apply to the iPraktikum project leaders as well.


