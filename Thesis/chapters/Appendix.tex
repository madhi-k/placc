\chapter{Appendix} \label{Appendix}

\begin{longtable}{|  p{0.2\textwidth}  |  p{0.7\textwidth} |}
\caption{Codes of the experience in the virtual setting}
\label{tab:table3}\\
\hline
\textbf{Interview code} & \textbf{Describing experience as PL} \\
\hline
\vspace{-0.5cm} I1 & 
   \begin{itemize}
    \vspace{-0.5cm} \item Some things got easier: finding a suitable time slot, students don't have the pressure to schedule in commuting time and can mix work and uni meetings, so a lot easier to organize that, Easier to reach students. Not necessary to communicate directly to get tasks done
    \vspace{-0.3cm} \item Downside: Social aspect. There used to be an onboarding and offboarding phase with the team before and after meetings. Team that filmed the trailer together really bonded together.
    \end{itemize} \\
\hline
\vspace{-0.5cm} I2 &
    \begin{itemize}
   \vspace{-0.5cm}  \item A lot more distance between PL and team members
    \vspace{-0.3cm} \item Everything got more static, less emotion involved
    \vspace{-0.3cm} \item They cannot talk directly to each other, which hinders the overall team dynamic.
    \vspace{-0.3cm} \item Need to work much more on communication
    \end{itemize}\\
\hline
\vspace{-0.5cm} I3 & 
    \begin{itemize}
   \vspace{-0.5cm}  \item Difficult to motivate the team
    \vspace{-0.3cm} \item Difficult convincing the students that this is something much bigger than a university course
    \end{itemize} \\
\hline
\vspace{-0.5cm} I4 & 
    \begin{itemize}
    \vspace{-0.5cm} \item You can’t see their reactions
    \vspace{-0.3cm} \item You are more involved in the meetings
    \end{itemize}
  \\
\hline
\vspace{-0.5cm} I5 & 
    \begin{itemize}
    \vspace{-0.5cm} \item The first big difference is the body language you are missing in virtual communication. Seeing their body language or seeing them as a whole person builds more trust, and you also learn about their personality, and that is missing.
   \vspace{-0.3cm}  \item harder to motivate people over the camera, when you can't engage, can't have small talk after the meeting
   \vspace{-0.3cm}  \item hard for the PL to trust the team and let loose a bit and not micromanage. Now I have to trust them blindly.
    \end{itemize}
\\
\hline
\vspace{-0.5cm} I6 & 
    \begin{itemize}
   	\vspace{-0.5cm} \item The teams that I have supervised were really good so this didn't seem to suffer from the virtual setting.
    \end{itemize}
   \\
\hline
\vspace{-0.5cm} I7 & 
    \begin{itemize}
    \vspace{-0.5cm} \item And for me as a PL it is important to get a feeling of what everyone is thinking and that is much easier to assess sitting next to each other. Because how someone speaks, gestures, body language are much easier to assess when you sit closely. And that is much more difficult to do in a virtual setting. 
    \end{itemize}
    \\
\hline
\vspace{-0.5cm} I8 & 
    \begin{itemize}
    \vspace{-0.5cm} \item It's been pretty good so far.
    \end{itemize}
    \\
\hline
\vspace{-0.5cm} I9 &
    \begin{itemize}
   \vspace{-0.5cm} \item Each three semesters, in the beginning it was a bit weird and difficult because a lot of information gets lost if you can’t see how they behave and I think that, if you are in person, sometimes people next to each other have more interactions with each other, and now one person is talking and the others are just listening. This reduces the level of “getting in touch” with each other and get closer faster.
    \vspace{-0.3cm} \item I think is that people get distracted much more easily in the virtual setting
    \vspace{-0.3cm} \item  in a person setting you also have more social interactions and can look at someone when you talk to them
      \vspace{-0.3cm} \item I have another point I find really challenging is that you do not have small talk anymore
    \end{itemize}
  \\
\hline
\vspace{-0.5cm} I10 & 
    \begin{itemize}
    \vspace{-0.5cm} \item I think it went better than expected. Both times if went really smoothly. But it wasn’t the same thing when being a coach, when we first learned the iPraktikum was going to be virtual. It really stressed me out at the beginning because it was quite a different experience form being a developer.
    \vspace{-0.3cm} \item Yeah that happens. I think I still am really close to the team, I think I am closer to the team than other project leads. Maybe because last semester my coach… I also tried at the beginning to act like a second coach. 
    \end{itemize}
     \\
\hline
\end{longtable}


\begin{longtable}{|  p{0.2\textwidth}  |  p{0.7\textwidth} |}
\caption{Codes of the casualty of remote meetings}
\label{tab:table4}\\
\hline
\textbf{Interview code} & \textbf{Describing experience as PL} \\
\hline
\vspace{-0.5cm} I1 & 
Students also become more detached. The ones used to virtual communication (e.g. gamers) are still the same, but other students make use / abuse of this setting to draw back from the meetings, switch off the camera. From a technological perspective, it is also easier to pretend you are present in the meeting, while you are actually doing something else. \\
\hline
\vspace{-0.5cm} I2 &
 Students scan attend meetings from everywhere, e.g. while riding a bike.

In a real meeting they could come up with some excuse to not attend the meeting, and if present, would be more attentive.

And this is another thing, people at their homes, I mean this doesn’t only hold true for the iPraktikum but for any level in any lecture, I think their mindset is not only into that work mode. Their mind is probably thinking about doing the groceries, feeding the dog or behind me my boyfriend or girlfriend is working. I feel you are not fully into the topic. This doesn't only hold for students but for me as well and I love to work from home general but then I’m fed up with working from home so I think some balance needs to be found.
 \\
\hline
\vspace{-0.5cm} I3 & 
   For me personally, I don't think that (being at home) impacts me that much, I think it makes switching off and relaxing easy. I think you get distracted at home but also at work. Maybe the atmosphere isn't right. 
   
   I also think that the person that would do that in an online meeting could also do that in a in-person meeting. Not to that extend and one of the reasons can be being at home, but one of them is that the actual impacts are mitigated by the virtual setting. 
  \\
\hline
\vspace{-0.5cm} I4 & 
    To be honest I haven't experienced something like this. Usually everyone has their own room. They are well prepared and just sit until the end of the meeting. There would be a problem if you share your flat or room, but for me there hasn't been any interference let's say. Wherever they were, it was like being in the same room, nobody else.
  \\
\hline
\vspace{-0.5cm} I5 & 
  If it's in some norm, it's okay, and with all these distractions and being at home it is more difficult to be in a working mindset. For me honestly, I skip some meetings because it is difficult to switch from doing something mundane, to solving world problems.
\\
\hline
\vspace{-0.5cm} I6 & 
   I haven't seen anything that was a major discomfort. It happens that people talk over one another because there is only one channel and a lot of latency involved. This is slightly annoying but not a big deal I think. Maybe one thing this year is that the sprint planning is a bit sluggish at the moment, that might be due to the virtual setting. It is harder to checkout if someone wants to take over a task, the coach has to wait longer for a response.
   \\
\hline
\vspace{-0.5cm} I7 & 
  Both, you can see people who are very comfortable not sitting together next to each other and being in the virtual environment where there are also people which go into the opposite direction. 
    \\
\hline
\vspace{-0.5cm} I8 & I am not sure. I am lucky enough to differ between a bedroom and an office room and leave the work for the office. I know though that this is difficult for most to differ between these two worlds and maybe the people can’t relax any more as easily. But most people separate that visually, whether it is with a virtual background or something else.  It's been pretty good so far.
    \\
\hline
\vspace{-0.5cm} I9 &
   It's difficult to say I think when I have a feeling that people are not listening the majority of the time so I mean there really more than 50 percent of the time they aren't listening they are doing something else and I have the feeling they just don't care about this meeting, because again, if you sit with other people you sit up straight most of the time trying to follow, even if they're not very interested they might just look somewhere else but they used to be present, where in these meetings then you can see when people disappear below their tables or just down in the chair. Of course, it's more comfortable at home but I mean it's but maybe part of the job is not being too comfortable.
  \\
\hline
\vspace{-0.5cm} I10 & 
  I think the first one two meetings are a bit awkward than what they would have been in real life. Just because you press a button and you get thrown into it and there is none of the feedback from the meetings you do in person. Then that kinda fades away a little bit. It gets harder to really estimate the other persons. Once you meet and really try to play games the first two meetings just to get more comfortable with each other. That really helped.
     \\
\hline
\end{longtable}

\begin{longtable}{|  p{0.2\textwidth}  |  p{0.7\textwidth} |}
\caption{Codes of being active in remote meetings}
\label{tab:table4}\\
\hline
\textbf{Interview code} & \textbf{Describing experience as PL} \\
\hline
\vspace{-0.5cm} I1 & 
Students also become more detached. The ones used to virtual communication (e.g. gamers) are still the same, but other students make use / abuse of this setting to draw back from the meetings, switch off the camera. From a technological perspective, it is also easier to pretend you are present in the meeting, while you are actually doing something else. \\
\hline
\vspace{-0.5cm} I2 &
 I think what is easier to solve in a real setting is that introverted people become more talkative overtime and over zoom at least from what I observed maybe some other project leaders too, is that the more introverted ones stay in their comfort zone and I don't know if psychologist study on that or if they would agree on that and if there is something to it, but in my experience again this online communication does not really develop your communication skills.
 \\
\hline
\vspace{-0.5cm} I3 & 
 That depends from team to team. In the beginning, the students were less comfortable, in the first few weeks. Everyone unmutes themselves when they have something to say, which is in their eyes pretty important. The virtual formal makes it more difficult for the students. But that changes over time when they have met each other already., I think it makes switching off and relaxing easy. I think you get distracted at home but also at work. Maybe the atmosphere isn't right.
  \\
\hline
\vspace{-0.5cm} I4 & 
    To be honest I haven't experienced something like this. Usually everyone has their own room. They are well prepared and just sit until the end of the meeting. There would be a problem if you share your flat or room, but for me there hasn't been any interference let's say. Wherever they were, it was like being in the same room, nobody else.
  \\
\hline
\vspace{-0.5cm} I5 & 
 Sometimes you have hyperactive people that jump into discussions, and in the virtual setting it is easier to calm them down, by talking over them maybe. And having a 5 minute check-in about the last week where everyone can say what they struggled with university-wise, opens up the communication. 
\\
\hline
\vspace{-0.5cm} I6 & 
  Not that I know of, at least not in a way that would disturb the meeting. So far, the people were focusing on the meeting and quite active.
   \\
\hline
\vspace{-0.5cm} I7 & 
 They don't say anything, they're not very active and even if you speak to them and say hey what do you think about ... and say their name and address questions to them to force them somehow to participate in those meetings. And that's kind of an anti pattern I experienced and that there are those two guys, the ones who are very communicative (most of them I think also have been communicative beforehand) I have to say but for those who are not that communicative I also don't think that they have been very communicative before. I guess they don't know really how to make up their voice during those meetings because there are these very dominant persons which speak all the time, and I bet, when you would give those people who a little bit more shy the chance to say something or space somehow that would work. I also have the feeling, as we never met in person, that you have kind of this distance between the people and if you have the distance and you're a shy person you might be afraid to say something wrong that's where the behavior of not saying something especially for shy people comes from.
    \\
\hline
\vspace{-0.5cm} I8 & Each three semesters, in the beginning it was a bit weird and difficult because a lot of information gets lost if you can’t see how they behave and I think that, if you are in person, sometimes people next to each other have more interactions with each other, and now one person is talking and the others are just listening. This reduces the level of “getting in touch” with each other and get closer faster.
    \\
\hline
\vspace{-0.5cm} I9 &
  It's difficult to say I think when I have a feeling that people are not listening the majority of the time so I mean there really more than 50 of the time they aren't listening they are doing something else and I have the feeling they just don't care about this meeting right because again if you sit with other people you sit up straight most of the time trying to follow, even if they're not very interested they might just look somewhere else but they used to be present, where in these meetings then you can see when people disappear below their tables or just down in the chair. Of course, it's more comfortable at home but I mean it's but maybe part of the job is not being too comfortable.

This is also one of the things that I have observing home really invading the meeting space whether it’s about positions like damaged air or on the bed or somewhere in the toilet but also in regards to this really comfortability they should have at home...

You would rarely see someone eating in on site meeting, we always brought some food for the middle of the sprint and now you can see people are just bringing their lunch basically during the meeting. But I have heard of this happening so in other key in many other things I mean eating and drinking maybe just one example.
  \\
\hline
\vspace{-0.5cm} I10 & 
It’s really hard because I was really lucky with my teams. Very motivated teams. Maybe it is even easier in the virtual setting to be active because someone is working in the evening, is on discord and wants to discuss something with the team. In an non-virtual setting you would be more likely to collaborate when you meet up in the work meetings. I wouldn’t say that it’s better, but it’s not worse.

Sometimes someone rings the door, or someone is in the train. I would not mind, but if the meeting was non-virtual, maybe they would not come at all to the meeting…

Totally, also because of the mask, and poor internet connection, you can’t really understand what they are saying. The train can be a bit tricky.

I don’t mind, if the team is being functional, I don’t care if it happens once or twice. Yeah that person then doesn’t speak that much during the meeting and tries to listen more, but happened only once. It didn’t even bother me. Because our meetings again are super smooth. And it was a process to get there.
     \\
\hline
\end{longtable}

\begin{longtable}[ht]{ p{0.25\textwidth}   p{0.1\textwidth}  p{0.45\textwidth} }
\caption{Codes of inappropriate behaviours}
\label{tab:table5}\\
\hline
\textbf{Behaviours} & \textbf{Interview} & \textbf{Code from the interview} \\
    \hline
    \multirow{2}{4cm}{Join meetings from random places} & I2 & On a specific instance: I noticed that he was more focused on his thoughts than on the team spirit or activities. They were always in a rush, what I did not notice is that the team was also not happy with his productivity and performance \\
    & I4 & This person doesn’t care about getting the most out of the iPraktikum and the meeting  \\
    & I6 & They don’t know how to be professional \\
    \hline
    \multirow{2}{4cm}{Being muted and not participating actively} & I7 & This person is not as experienced and are usually lacking in productivity \\
    & I2 & They are just shy/ introverted. You could think he would be one of the people that tags along. \\
    & I4 & They think this is just another lecture and they just say something in the standup table and then they just mute themselves and don't speak a word for the rest of the meeting. They just join to not be punished.  \\
 	& I9 & They are doing something else. I am concerned that this effects their productivity  \\
    \hline
     \multirow{2}{4cm}{Constantly switching the camera on and off} & I2 & It’s very confusing when they do this and it annoys me \\
    & I3 & having it off without a reason is a bit strange, gives me the feeling that they are not really paying attention, and that they have something else going on that they need to pay attention to and don't want to show, and that def. effects the way I see someone in a meeting.  \\
    & I7 & It’s definitely an anti-pattern, I don’t like that \\
 	& I9 & This shouldn’t happen, this is a meeting, and all should commit and respect each other  \\
    \hline
    \multirow{2}{4cm}{Engaging in physical movements} & I3 & ...or walking around it's worse.\\
    & I7 & This is rude and unprofessional \\
    \hline
    \multirow{2}{4cm}{Engaging in conversation with other people in their physical room} & I3 & Inappropriate and disinterest in the meeting \\
    & I5 & That is totally unacceptable, you are in a team meeting, and should be 100 percent focused there, even for me, I let everyone know that they should not bother me or distract me while in a team meeting. \\
    & I7 & Disrespectful towards everyone in the meeting. This is distracting to the participants of the meeting, me included, and steals our attention from the meeting.  \\
 	& I9 & It is unprofessional as the private life should not interfere with the meeting. Other team members might not feel being taken seriously.  \\
    \hline
     \multirow{2}{4cm}{Agreeing to “too many” things in the meeting or use empty phrases but are not actually delivering} & I3 & They were being too optimistic, and asking about the consequences of not achieving something on time, and that gave me the impression they were lazy. \\
    & I1 & It feels like I’m interacting with a clam \\
    & I4 & They are not interested in the project  \\
 	& I5 & It’s not in their culture to ask for support \\
    \hline
     \multirow{2}{4cm}{Actively doing something beside the meeting on their computer} & I4 & I would consider this rude and are not eager to bring the project forward \\
    & I3 & I think when you are actually doing something in you laptop is less inappropriate (than communicating with others in the room) \\
 	& I9 & Most of the time I don't attribute that to the performance but in some cases I do, when I see that some person is doing this all the time.\\
    \hline
     \multirow{2}{4cm}{Slouching and looking quite sleepy} & I7 & They are probably bored and are not really attending \\
    & I9 & They are not in the mood and are quite inappropriate. They don’t care about getting out of the comfort zone of home. They feel too comfortable at home. \\
    \hline
    \multirow{2}{4cm}{Constantly interacting with their phone. They look at it, take phone calls…} & I2 & They might be waiting for something important \\
    & I4 & It is obvious they are not as interested in what everyone else is saying, or in the project at all. I would def draw the conclusion that they don't really want to pay attention, they think it is not important what is being said. \\
 	& I5 & Checking the phone is something that would not normally happen in a meeting \\
    \hline
    \multirow{2}{4cm}{Having a stack of dishes waiting to be cleaned and a pile of clothes behind} & I5 & This person is quite disorganized and that translates to their work \\
    & I7 & This is disrespectful to the participants, especially in formal meetings \\
    \hline
    \multirow{2}{4cm}{A person doesn’t make themselves visible in the meeting (because they sit in the dark, the camera is in a weird position)} & I5 & It is a bit weird and they at least looking in the direction of the camera. \\
    & I9 & They are not really being attentive. They don’t want to participate in the activities and don’t want to show when they are doing something else\\
 	& I10 & They couldn’t do better \\
    \hline
    \multirow{2}{4cm}{Facing problems connecting their devices or seems reluctant to do so} 
    & I2 & Not knowing how to connect to zoom, or screen share, which has no excuse for a computer scientist. \\
    & I3 & If it takes for example long for you to connect. But we are also human and that is excusable if you directly communicate that. In the meeting itself it does leave a bad mark if you are screen sharing and you have 20 tabs open. Or if you need to tell a person three times that they need to share their screen for a demo and still doesn't happen then it does leave a bad impression. I think it shows that you are not interested, probably opened that just before sharing the screen. \\
    & I4 & Yeah that's definitely something annoying when some people open up their laptop one minute before and then they say they still have to update zoom or install something... It's of course frustrating when people come unprepared and don't have everything ready and so on. There is always one person in the meeting that is unprepared.. showing that this doesn't have any priority for them. \\
    \hline
     \multirow{2}{4cm}{Leave early from a meeting, being late at a meeting or not attending:} & I1 & Unaware that the iPraktikum goes beyond the scope of a practical course \\
      & I2 & Coming late to meeting is quite annoying \\
      & I3 & Of course there is rude behaviour, like ... or joining late, like in normal in-person meeting \\
      & I4 & ... this person doesn't want to be there because maybe always wants to jump off at the ending \\
    \hline
     \multirow{2}{4cm}{Having audio issues and a bad quality of the sound} & I2 & I know that it is hard, and I know that not everyone can afford having a good mac, but this is your profession. This is your work environment and need to invest in it. \\
    & I4 & I know sometimes it's hard to listen to, it hurts my ears if I have to listen to that for two hours.. he's always then shouting and so on... a bad microphone quality can def hurt. \\
    \hline
     \multirow{2}{4cm}{A student constantly interrupts/talks over the others:} & I7 & That’s very unprofessional and rude \\
    & I9 & They want to dominate the situation and not make the others heard \\
    \hline
     \multirow{2}{4cm}{not looking into the direction of the camera} & I3 & They should at least look into the direction of the camera (interview 3), otherwise more social cues are missing \\
    \hline
     \multirow{2}{4cm}{Walking out of the meeting unannounced:} & I3 & Of course there is rude behaviour, like exit the zoom meeting without saying anything and no one understands what happened there... \\
     & I8 & Rude \\
    \hline
     \multirow{2}{4cm}{Directly communicating with the customers:} & I1 & Lack of professional communication, communication streams are les professional \\
    \hline
     \multirow{2}{4cm}{Not agreeing to using certain tools} & I2 & I think there are a few ruder things here and there. E.g. I don't use tool x and y because whatever reason. This is not great for everybody, blockading a tool makes it a bit harder on the team. \\
    \hline
\label{tab:multicol}
\end{longtable}

\begin{longtable}{|  p{0.4\textwidth}  |  p{0.5\textwidth} |}
\caption{Questionnaire}
\label{tab:table6}\\
\hline
\textbf{Survey question} & \textbf{Statements or Attributions} \\
\hline
\multirow{4}{6cm}{Q1: If one of your team members would often join meetings from their phone, (e.g. while running errands, commuting) would you think:} & 
    \hspace{0.3cm} A1.1: They are not actually attentive and should schedule more time for the meetings  \\
    & \hspace{0.3cm} A1.2: This person might not be involved in the team spirit \\
    & \hspace{0.3cm} A1.3: This person isn't particularly interested in getting the most out of the iPraktikum and the meeting \\
    &  \hspace{0.3cm} A1.4: I would personally consider this quite unprofessional \\
\hline
\multicolumn{2}{|c|}{Would some other statements better describe your impression?} \\
\hline
\multirow{5}{6cm}{Q2: A person in your team is mostly just listening during the meetings and is muted almost all the time:} &
   \hspace{0.3cm} A2.1: This person is probably not as experienced as the others \\
    & \hspace{0.3cm} A2.2: I would assume they are doing something else in parallel\\
    & \hspace{0.3cm} A2.3: They are just shy/introverted \\
    & \hspace{0.3cm} A2.4: They consider the iPraktikum the typical uni lecture and are just there to not absent\\
   &  \hspace{0.3cm} A2.5: They are likely facing impediments with their tasks/ struggling with productivity \\
 \hline
\multicolumn{2}{|c|}{Would another statement be more fitting?} \\
\hline
\multirow{4}{6cm}{Q3: But then you realize that a person is actively doing something besides the meeting on their computer (seems focused on another task, are typing):} & 
    \hspace{0.3cm} A3.1: I would consider this impolite towards the participants of the meeting \\
   & \hspace{0.3cm} A3.2: The advancement of the project might not be a priority\\
   & \hspace{0.3cm} A3.3: The project itself is seemingly not appealing to them \\
   & \hspace{0.3cm} A3.4: They probably just want to pretend to be in the meeting, but not put in the work\\
\hline
\multicolumn{2}{|c|}{Maybe something else applies?} \\
\hline
\multirow{3}{6cm}{Q4: One of the team members constantly turns their camera on and off to do something meanwhile:} & 
    \hspace{0.3cm} A4.1: I would assume they are not that committed to the project and teamwork \\
    & \hspace{0.3cm} A4.2: It’s confusing to me why they do this and a bit disrespectful\\
    & \hspace{0.3cm} A4.3: They are possibly not actively listening and don’t want to show what they are doing instead\\
\hline
\multicolumn{2}{|c|}{Would another statement be more suitable to you?} \\
\hline
\multirow{4}{6cm}{Q5: A person enters/is in the room of one of the participants and they might even engage in conversation:} & 
    \hspace{0.3cm} A5.1: This might be offensive towards the participants in the meeting \\
    & \hspace{0.3cm} A5.2: To me, it looks like they are not focused on the meeting and get distracted easily \\
    & \hspace{0.3cm} A5.3: This might indicate that they are not contributing enough to the team \\
   & \hspace{0.3cm} A5.4: They probably easily mix private and professional life \\
\hline
\multicolumn{2}{|c|}{If none of these statements apply, what would?} \\
\hline
\multirow{4}{6cm}{Q6: A student is agreeing to “too many” things in the meeting or use empty phrases, but are not actually delivering:} & 
   	\hspace{0.3cm} A6.1: They might be lacking the skills to keep up to their promises \\
   	& \hspace{0.3cm} A6.2: Not getting enough feedback makes me think this is not a person you can easily deal with \\
   	& \hspace{0.3cm} A6.3: They might not be really interested in the project and are just pretending to agree \\
   & \hspace{0.3cm} A6.4: It’s not in their culture/personality to reply or ask for support \\
\hline
\multicolumn{2}{|c|}{Other better-fitting statement maybe?} \\
\hline
\multirow{3}{6cm}{Q7: A person in the team is slouching and looking quite sleepy…}
 & 
    \hspace{0.3cm} A7.1: Perhaps, they are bored and are not really attending \\
   &  \hspace{0.3cm} A7.2: This "sluggishness" probably translates into their work \\
    & \hspace{0.3cm} A7.3: They might consider it unnecessary to get out of the comfort zone of their home and make an extra effort \\
\hline
\multicolumn{2}{|c|}{Another opinion maybe?} \\
\hline
\multirow{4}{6cm}{Q8: One team member is constantly interacting with their phone. They look at it, check notifications, take phone calls, even type…}
 & 
    \hspace{0.3cm} A8.1: They could be into social media or communication apps \\
    & \hspace{0.3cm} A8.2: This is quite an immature thing to do in meetings \\
    & \hspace{0.3cm} A8.3: They are not interested in the discussion and in what everyone else is saying  \\
    &  \\
\hline
\multicolumn{2}{|c|}{Other reasons?} \\
\hline
\multirow{2}{6cm}{Q9: A student has a stack of dishes waiting to be cleaned or a pile of clothes behind. You think:} &
   \hspace{0.3cm} A9.1: This person might be disorganized in their work too \\
    & \hspace{0.3cm} A9.2: This could be disrespectful to the participants, especially in formal meetings 
  \\
  &  \\
\hline
\multicolumn{2}{|c|}{Would another statement be more suitable} \\
\hline
\multirow{3}{6cm}{Q10: A person doesn’t make themselves visible in the meeting (not because their camera is off, but because they sit in the dark, the camera is in a weird position):} & 
    \hspace{0.3cm} A10.1: They don’t want to show their facial expressions and reactions \\
    & \hspace{0.3cm} A10.2: This could make me suspicious of their performance \\
    & \hspace{0.3cm} A10.3: They might not be the most engaged and cooperative in the team
     \\ &  \\ & \\
\hline
\multicolumn{2}{|c|}{If you disagree, would there be a more realistic perception?} \\
\hline
\multirow{4}{6cm}{Q11: One of the participants is facing problems connecting his devices or seems reluctant to do so. You probably also see other things open on their screen:} & 
    \hspace{0.3cm} A11.1: The iPraktikum is not a priority to them this semester \\
    & \hspace{0.3cm} A11.2: They are probably just opening the page before having to share the screen \\
    & \hspace{0.3cm} A11.3: They might not be actively sharing so they don't have to show their actual progress
     \\
\hline
\multicolumn{2}{|c|}{Other reasons this might happen?} \\
\hline
\multirow{3}{6cm}{Q12: A student is having audio issues and the quality of the sound is very bad.} & 
    \hspace{0.3cm} A12.1: They should consider switching or investing in a more professional setting \\
    & \hspace{0.3cm} A12.2: It might hurt the quality of the meeting and it renders the person less approachable \\
   &  \hspace{0.3cm} A12.3: It makes me think that this person is not motivated  \\
\hline
\multicolumn{2}{|c|}{Another opinion you would have?} \\
\hline
\multirow{3}{6cm}{Q13: A student constantly interrupts/talks over the participants of a meeting:} & 
    \hspace{0.3cm} A13.1: They want to impose their opinion on others \\
    & \hspace{0.3cm} A13.2: That don't want the other members to be heard which is quite improper \\
    & \hspace{0.3cm} A13.3: This is personally even rude  \\
\hline
\multicolumn{2}{|c|}{Or maybe something else?} \\
\hline
\multirow{2}{6cm}{Q14: Having the camera on, the student engages in physical movements, like walking around or performing some physical exercise:} & 
    \hspace{0.3cm} A14.1: This is inconsiderate towards the other participants \\
    & \hspace{0.3cm} A14.2: To me, it shows an unwillingness to adapt to a professional setting \\
\hline
\multicolumn{2}{|c|}{If disagreeing, how would you consider this behaviour instead?} \\
\hline
\multirow{7}{6cm}{Q15: To what extent do the following statements apply to you?} & 
    \hspace{0.3cm} A15.1: I trust other people almost immediately \\
    & \hspace{0.3cm} A15.2: It's difficult for me to trust other people actually \\
    & \hspace{0.3cm} A15.3: I usually get concerned about the well-being of this person, and instruct the coach to look into that \\
    & \hspace{0.3cm} A15.4: It is not my responsibility to make others feel better of look after everyone \\
    & \hspace{0.3cm} A15.5: I am quite sharp-tongued when I get back at others \\
    & \hspace{0.3cm} A15.6: I usually try not to contradict others \\
    & \hspace{0.3cm} A15.7: I state my opinion several times since I know which is the right thing to do
     \\
\hline
\end{longtable}


\begin{longtable}{  p{0.3\textwidth}   p{0.12\textwidth}  p{0.1\textwidth}  p{0.1\textwidth}  p{0.1\textwidth}  p{0.1\textwidth} }
 \caption[Attribution Question Frequencies]{Frequencies of attribution responses} \\
	\bottomrule
    Statement  & Strongly Disagree & Disagree & Neutral & Agree & Strongly Agree \\
	\bottomrule
   A1.1 & 0\% & 0\%  & 11.1\%  &   44.4\%  & 44.4\%
   \\
   A1.2  & 0\% & 11.1 \% & 22.2\% & 44.4\% & 22.2\%
     \\
     A1.3  & 0\% & 11.1\% & 0\% & 55.6\% &  33.3\% 
    \\
     A1.4 & 0\% & 11.1\% & 22.2\% &  22.2\% & 44.4\%
      \\
      \hline
      A2.1  & 11.1\% & 22.2\% & 55.6\% & 11.1\% & 0\%
      \\
      A2.2  & 0\% & 44.4\% & 33.3\% &  22.2\% & 0\%
     \\ 
    A2.3  &  0\% & 0\% & 11.1\% & 66.7\%  &  22.2\%   
    \\ 
    A2.4  &  11.1\% & 11.1\%  & 55.6\%  &   22.2\%  & 0\%  
    \\ 
    A2.5  &  11.1\% & 22.2\%  & 44.4\%  &   22.2\%  & 0\%  
    \\
    \hline
      A3.1 & 0\% & 0\% & 0\% & 44.4\% & 55.6\% 
      \\
      A3.2 & 0\% & 0\% & 0\% &  77.8\% & 0\%
     \\ 
    A3.3 & 0\% & 22.2\% & 44.4.1\% & 33.3\%  &  0\%   
    \\ 
    A3.4 & 0\% & 22.2\%  & 44.4\%  & 11.1\%  & 22.2\% 
    \\
    \hline
      A4.1 & 0\% & 11.1\% & 55.6\% & 33.3\% & 0\% 
      \\
      A4.2 & 0\% & 0\% & 11.1\% &  44.4\% & 44.4\%
     \\ 
    A4.3 & 0\% & 0\% & 22.2\% & 66.7\%  &  11.1\%  
    \\
    \hline
      A5.1 & 0\% & 11.1\% & 22.2\% & 33.3\% & 33.3\% 
      \\
      A5.2 & 11.1\% & 11.1\% & 44.4\% &  33.3\% & 0\%
     \\ 
    A5.3 & 0\% & 66.7\% & 33.3\% & 0\%  &  0\%   
    \\ 
    A5.4 & 11.1\% & 11.1\%  & 44.4\%  & 22.2\%  & 0\% 
    \\
    \hline
      A6.1 & 0\% & 0\% & 0\% & 100\% & 0\% 
      \\
      A6.2 & 0\% & 22.2\% & 44.4\% &  33.3\% & 0\%
     \\ 
    A6.3 & 0\% & 44.4\% & 22.2\% & 22.2\%  &  11.1\%   
    \\ 
    A6.4 & 0\% & 0\%  & 44.4\%  & 55.6\%  & 0\% 
    \\
     \hline
      A7.1 & 0\% & 33.3\% & 11.1\% & 55.6\% & 0\% 
      \\
      A7.2 & 11.1\% & 22.2\% & 44.4\% & 22.2\% & 0\%
     \\ 
    A7.3 & 0\% & 22.2\% & 44.4\% & 33.3\%  &  0\%   
    \\ 
    \hline
      A8.1 & 0\% & 0\% & 33.3\% & 33.3\% & 33.3\% 
      \\
      A8.2 & 0\% & 0\% & 44.4\% & 55.6\% & 0\%
     \\ 
    A8.3 & 0\% & 0\% & 33.3\% & 66.7\%  &  0\%   
    \\ 
    \hline
      A9.1 & 11.1\% & 33.3\% & 33.3\% & 22.2\% & 0\% 
      \\
      A9.2 & 11.1\% & 33.3\% & 22.2\% & 22.2\% & 11.1\%
     \\ 
     \hline
      A10.1 & 0\% & 55.6\% & 33.3\% & 11.1\% & 0\% 
      \\
      A10.2 & 22.2\% & 22.2\% & 33.3\% & 22.2\% & 0\%
     \\ 
    A10.3 & 11.1\% & 33.3\% & 22.2\% & 22.2\%  &  11.1\%   
    \\ 
     \hline
      A11.1 & 0\% & 33.3\% & 33.3\% & 33.3\% & 0\% 
      \\
      A11.2 & 0\% & 11.1\% & 66.7\% & 22.2\% & 0\%
     \\ 
    A11.3 &  0\% & 33.3\% & 44.4\% & 22.2\% & 0\%   
    \\ 
     \hline
      A12.1 & 0\% & 11.1\% & 33.3\% & 44.4\% & 11.1\% 
      \\
      A12.2 & 0\% & 0\% & 0\% & 44.4\% & 55.6\%
     \\ 
    A12.3 & 22.2\% & 22.2\% & 44.4\% & 11.1\% & 0\%   
    \\ 
    \hline
      A13.1 & 0\% & 22.2\% & 0\% & 66.7\% & 11.1\% 
      \\
      A13.2 & 0\% & 33.3\% & 33.3\% & 33.3\% & 0\%
     \\ 
    A13.3 & 0\% &  0\% & 33.3\% & 44.4\% & 22.2\%   
    \\ 
    \hline
      A14.1 & 0\% & 22.2\% & 0\% & 55.6\% & 22.2\% 
      \\
      A14.2 & 0\% & 11.1\% & 0\% & 66.7\% & 22.2\%
     \\ 
    \bottomrule 
    \label{tab:AttributionResponsesFrequencies}  
\end{longtable}

\begin{longtable}{ | p{0.25\textwidth} |  p{0.3\textwidth} | c | c | c | c | c | c | c | c |}
\caption{Persona allocation} \\
% \centering
    %\begin{tabular}{clcccccccccc}
    \hline
        \textbf{Behaviour} & \textbf{Responses} & \multicolumn{7}{c}{\textbf{Personas}} \\
        \hline
        & & \rot{The Unprofessional} \vline & \rot{Ego is the enemy} \vline & \rot{L'Étranger} \vline & \rot{The Loner} \vline
        & \rot{The Underperformer} \vline & \rot{Hiding and not seeking} \vline & \rot{Distraction Monster} \vline \\
        \midrule
        \multirow{4}{4cm}{{1. One of your team members often joins meetings from their phone, (e.g. while running errands, commuting)}}
        & They are actually not attentive and must schedule more time for the meetings
                     &  &   &  *  &   &   &   &     \\ 
        & This person might not be involved in the team spirit
                       &  &  &  & * &  &  &    \\
        & This person isn't particularly interested in getting the most out of the iPraktikum and the meeting  &  &   & * &  &  &   &    \\
        &I would personally consider this quite unprofessional               & * &   &   &   &   &   &     \\
        \hline
        \multirow{5}{4cm}{{2. A person in your team is mostly just listening during the meetings and is muted almost all the time}}
        & This person is probably not as experienced as the others 
                    &  &   &   &   & *  &   &    \\
        & I would assume they are doing something else in parallel
                       &  &  &  &  &  & * &  \\
        & They are just shy/introverted
                      &  &   &    & *  &  &  &    \\
        & They consider the iPraktikum the typical uni lecture and are just there to not absent
        				&   &   & * &   &  &   &   \\
        & They are likely facing impediments with their tasks/ struggling with productivity
                      &  &   &   &   &  * &   &   \\
       \hline
        \multirow{5}{4cm}{{3. But then you realize that a person is actively doing something besides the meeting on their computer (seems focused on another task, are typing)}}
        & I would consider this impolite towards the participants of the meeting
                    &  & *  &   &   &   &   &    \\
        & The advancement of the project might not be a priority
                       &  &  & * &  &  &  &   \\
        & The project itself is seemingly not appealing to them 
                      &  &   & *  &   &  &  &  \\
        & They probably just want to pretend to be in the meeting, but not put in the work
        				&  &  &  &  & * &   &   \\
       \hline
        \multirow{3}{4cm}{{4. One of the team members constantly turns their camera on and off to do something meanwhile: }}
        & I would assume they are not that committed to the project and teamwork 
                    &   &   &   &   &  *  &   &  \\
        & It’s confusing to me why they do this and a bit disrespectful
                       &  & * &  &  &  &  &  \\
        & They are possibly not actively listening and don’t want to show what they are doing instead 
                      &   &   &   &   &  & * &   \\
        \hline
        \multirow{5}{4cm}{{5. A person enters/is in the room of one of the participants and they might even engage in conversation: }}
        & This might be offensive towards the participants in the meeting
                    &  & *  &   &   &   &   &    \\
        & To me, it looks like they are not focused on the meeting and get distracted easily
                       &  &  &  &  &  &  & *  \\
        & This might indicate that they are not contributing enough to the team 
                      &  &   &   &   & * &  &    \\
        & They probably easily mix private and professional life 
        				& * &  &  &  &  &   &   \\
        \hline
        \multirow{5}{4cm}{{6. A student is agreeing to “too many” things in the meeting or use empty phrases, but are not actually delivering: }}
        & They might be lacking the skills to keep up to their promises
                    &   &   &   &   & *  &   &    \\
        & Not getting enough feedback makes me think this is not a person you can easily deal with
                       &  &  &  & *  &  &  &   \\
        & They might not be really interested in the project and are just pretending to agree
                      &  &   & *  &   &  &  &    \\
        & It’s not in their culture/personality to reply or ask for support
        				&  &  &  & * &  &   &   \\
        \hline
        \multirow{5}{4cm}{{7. A person in the team is slouching and looking quite sleepy… }}
        & Perhaps, they are bored and are not really attending
                    &  &   &   &   &   &   & *   \\
        & This "sluggishness" probably translates into their work
                       &   &  &   &   & * &   &   \\
        & They might consider it unnecessary to get out of the comfort zone of their home and make an extra effort
                      &  &   & *  &   &  &  &    \\
       \hline
        \multirow{5}{4cm}{{8. One team member is constantly interacting with their phone. They look at it, check notifications, take phone calls, even type… }}
        & They could be into social media or communication apps
                    &  &   &   &   &   &   & *   \\
        & This is quite an immature thing to do in meetings
                       & * &  &  &  &  &  &   \\
        & They are not interested in the discussion and in what everyone else is saying 
                      &  &   & * &  &  &  &    \\
        \hline
        \multirow{5}{4cm}{{9. A student has a stack of dishes waiting to be cleaned or a pile of clothes behind. You think: }}
        & This person might be disorganized in their work too
                    &  &   &   &   & *  &   &    \\
        & This could be disrespectful to the participants, especially in formal meetings 
                       &  & *  &  &  &   &  &   \\
        \hline
        \multirow{6}{4cm}{{10. A person doesn’t make themselves visible in the meeting (not because their camera is off, but because they sit in the dark, the camera is in a weird position): }}
        & They don’t want to show their facial expressions and reactions
                    &  &   &   &   &   & *  &    \\
        & This could make me suspicious of their performance
                       &  &  &  &  & * &  &   \\
        & They might not be the most engaged and cooperative in the team
                      &  &   &   & *  &  &  &    \\
           &  &  &   &   &   &  &  &    \\
           &  &  &   &   &   &  &  &    \\
           &  &  &   &   &   &  &  &    \\
        \hline
        \multirow{3}{4cm}{{11. One of the participants is facing problems connecting his devices or seems reluctant to do so. You probably also see other things open on their screen: }}
        & The iPraktikum is not a priority to them this semester
                    &  &   & *  &   &   &   &   \\
        & They are probably just opening the page before having to share the screen
                       &  &  & * &  &  &  &   \\
        & They might not be actively sharing so they don't have to show their actual progress
                      &  &   &   &   &  & * &    \\
        \hline
        \multirow{5}{4cm}{{12. A student is having audio issues and the quality of the sound is very bad. }}
        & They should consider switching or investing in a more professional setting
                    & * &   &   &   &   &   &   \\
        & It might hurt the quality of the meeting and it renders the person less approachable
                       &  &  &  & * &  &  &  \\
        & It makes me think that this person is not motivated 
                      &  &   &   & * &  &  &    \\
       \hline
        \multirow{5}{4cm}{{13. A student constantly interrupts/talks over the participants of a meeting: }}
        & They want to impose their opinion on others
                    &  & *  &   &   &   &   &    \\
        & They don't want the other members to be heard which is quite improper
                       &  & * &  &  &  &  &   \\
        & This is personally even rude
                      &  & *  &   &   &  &  &    \\
        \hline
        \multirow{5}{4cm}{{14. Having the camera on, the student engages in physical movements, like walking around or performing some physical exercise: }}
        & This is inconsiderate towards the other participants
                    &  & *  &   &   &   &   &    \\
        & To me, it shows an unwillingness to adapt to a professional setting
                       & * &  &  &  &  &  &   \\
         &  &  &   &   &   &  &  &    \\
         &  &  &   &   &   &  &  &    \\
         &  &  &   &   &   &  &  &    \\
        \bottomrule
   % \end{tabular}
\end{longtable}
