\chapter{Interviews}

Being the first phase of the thesis, the process of defining the interview questions, preparing the setting, and afterwards, analysing the findings are crucial to the phases following, and to the thesis in general. This chapter aims at elaborating in more detail the steps required for the conduct of the interviews (\ref{Setting} and \ref{DesignProcess}). The findings from the interviews can be found in \ref{Generic}, where the more generic aspects are clustered together and in \ref{Behaviours}, where we will look at the most common inappropriate behaviours PLs have faced, and what their perception of these behaviours was. Section \ref{Analysis} describes how these findings fit into the purpose of this thesis and how they contribute to the next steps. Lastly, we will take a look into the limitations and validity of the interviews in \ref{Limitations}.

\section{Setting}\label{Setting}

In order to foster flexibility, but also a certain degree of openness and freedom, the semi-structured interview method was chosen as a first phase for this thesis. Such a method has the potential to be more revealing than questionnaire surveys \cite{Barriball1994}, it gives the opportunity to explore more than just behaviours, but also attitudes, reactions and reflections. 

The semi-structured interview also gives interviewers some choice in the wording to each question but also in the use of probes \cite{Hutchinson1992}. Probing, in particular, can be a great tool for ensuring reliability of the data as it allows for the clarification of interesting and relevant issues raised by the respondents  \cite{Hutchinson1992}, provides opportunities to explore sensitive issues \cite{NayBrock1984} and can elicit valuable and complete information.

To structure the interview, the approach and suggestions described by DeJonckheere and Vaughn \cite{Dejonckheere2019} are best suitable for the environment, in which the interviews are to be conducted. The steps described in their paper will be elaborated in the sections to come.

\subsection{Objectives}

Initially, it was important to determine the purpose and scope of the interviews. Considering the research questions of this thesis, the interviews would contribute to answering RQ1: \textit{Which are the biases project leaders exhibit in the perception they have on the team members, considering the virtual scenario?}

Goal of this RQ is to identify behaviours or situations that might drive the project leader into dispositional attribution, and what these attributions are. The foundation would be inappropriate behaviours according to each project leader, which would then facilitate the identification of biases. The scope of the interviews would be to also gather more insights on general issues, and investigate how they could be used to construct a potential Persona.

\subsection{Participants}

The participants of the interviews were all project leaders in the iPraktikum of Summer Semester 2021. In this case study and environment, the project leader acted as a product owner in the team, responsible for communicating the project goal and defining and ensuring the realization of the backlog items in the product backlog. Considering we are conducting interviews within the framework of a case study of the iPraktikum21, and the participants were conveniently located and easily accessed, the sampling strategy corresponds to convenience sampling \cite{Galloway2005}. 

To provide a comprehensive overview, 10 project leaders were selected to be contacted, from different backgrounds and experience levels. All agreed to participate in the interview, reaching a response rate of 100\%. When approaching the project leaders, attention was given to the right introduction of the topic. Oppenheim \cite{Oppenheim1992} has stated that perhaps, the most important determinant both of response rate and of the quality of the responses is the subject's motivation, in our case the project leaders. Therefore, it was crucial to adequately describe the motivation of the topic, and how the exploration of such topics can help future project leaders. 

\subsection{Interviews' conduct}

To provide the right atmosphere and to establish a more personal connection, it was crucial to have a previous knowledge of each PLs' experiences in the iPraktikum. This knowledge would also add naturality to the interview and allow the interviewer to adapt accordingly in certain situations. It is also important to guarantee the interviewees privacy in the process, therefore, it was decided that no actual names and no companies would be transcribed and shared. 

The contacting of the interviewees and the conduct of the interviews themselves was semi-sequential, to minimize collision between the time slots, give opportunity to arrange the interview in less intensive weeks and to prevent interview fatigue. The interviewer created a doodle with their free timeslots, and shared it with the interviewees if needed, in order to accelerate the process of scheduling an appointment. The responses were generally fast, except 2 interviewees, which took a week to answer.

The contacting of the interviewees and the conduct of the interviews themselves was semi-sequential, to minimize collision between the time slots, give opportunity to arrange the interview in less intensive weeks and to prevent a mixture of the interviews. The interviewer created a doodle with their free timeslots, and shared it with the interviewees if needed, in order to accelerate the process of scheduling an appointment.

The participants were contacted via RocketChat and Slack, the messaging applications of choice used in the iPraktikum. Because of the times’ COVID regulations, but also due to the practicality of virtual meeting tools such as Zoom, 9 out of 10 interviews were conducted virtually, and only one interview in-person. Zoom’s built-in feature to record meetings was used to record both the video transmission and the audio. The in-person interview was audio-recorded via iPhone’s “Voice memos” App. The location of the virtual interviews did not play a role in the process of interviewing, except for one instance in which the internet connection was unstable from time to time, but still good enough to maintain steady communication with the interviewee. The in-person interview took place at the interviewee’s office, providing again an optimal environment.

The interviews were conducted privately, at least from the visibility range of Zoom. Interruptions of the interviews occurred in very rare cases, where a notification would temporarily distract the interviewee, or when a short side task needed to be performed. At the end of each interview, the interviewees were all invited to provide follow-up information, if they would encounter a situation that would contribute to the thesis. During the interviews themselves, no interviewee refused to answer a specific question, seemed apprehensive, or refused to have their interview recorded.

More details about the date, time, duration and data collection model of the the interviews can be found in ~\autoref{tab:interviewDetails}.

\begin{table}[h]
  \caption[Semi-structured interviews' details]{Semi-structured interviews' details}\label{tab:interviewDetails}
  \centering
  \begin{tabular}{ p{0.1\textwidth} p{0.15\textwidth}  p{0.1\textwidth} p{0.1\textwidth} p{0.40\textwidth}}
    \toprule
    Code Name  & Date & Time & Duration & Data Collecting Mode \\
    \midrule
     I1  &  27.05.2021 &  12:30 & 71 min & Voice and video recording via Zoom
     \\
      I2  &  05.06.2021 &  19:00 & 67 min & Voice and video recording via Zoom
     \\
      I3  &  07.06.2021 &  17:00 & 50 min & Voice and video recording via Zoom
      \\
      I4  &  09.06.2021 &  10:00 & 63 min & Voice and video recording via Zoom
      \\
      I5  &  09.06.2021 &  11:30 & 60 min & Voice and video recording via Zoom 
     \\
      I6  &  09.06.2021 &  15:00 & 23 min & Voice and video recording via Zoom
     \\
      I7  &  15.06.2021 &  15:00 & 54 min & Voice and video recording via Zoom
     \\
      I8  &  16.06.2021 &  14:45 & 44 min & Voice and video recording via Zoom 
   \\
    I9  &  25.06.2021 &  10:00 & 40 min & Voice recording via Voice Memos
    \\
     I10  &  30.06.2021 &  09:00 & 34 min & Voice and video recording via Zoom
     \\  
    \bottomrule
  \end{tabular}
\end{table}

\section{Design Process}\label{DesignProcess}

As a semi-structured interview, a guide with questions needed to be designed in order to ensure that the scope and purpose of the interview are achieved. Three types of questions were composed: guiding questions including "grand tour" questions, core questions and planned and unplanned follow-up questions.

The grand-tour questions aimed at making the interviewee comfortable with the interviewer and the interview itself. They are usually generic questions that give the opportunity to give personal insights and allow the interviewee to get in touch with their thoughts and their experiences before the more in-depth questions are addressed. 

These more in-depth questions, also known as core questions, directly tackle the goal of the thesis, and are arguably the most important questions of the interview. They served the identification of behaviours that lead to the emergence of attribution bias, and therefore needed to be structured intentionally. In the initial version of the interview, the interviewee was expected to spontaneously mention such behaviours, therefore the core questions revolved around identifying the bias attached to the above-mentioned behaviours. Follow-up questions were also planned for the case an answer would take a certain direction. 

The interview was planned to be concluded with a question that summarizes the interview, but also potentially entice more conversations. 

Since the quality and efficacy of an interview cannot be assessed during its design, it is crucial to pilot test the interview, and adjust accordingly. The initial interview guide is presented in~\autoref{tab:interview}, and it also includes the goal for each of the questions.

\begin{table}[htpb]
  \caption[First draft of interview questions]{First draft of interview questions}\label{tab:interview}
  \centering
  \begin{tabular}{l p{0.4\textwidth}  l p{0.4\textwidth} l p{0.4\textwidth}}
    \toprule
      & Question group & Goal \\
    \midrule
     Grand-tour  & 
     How has the experience as a project leader in the remote setting been? 

What do you think has been the most difficult thing to handle in the interaction with the team?

Do you think the team is more uncomfortable or comfortable in the virtual setting? How would you tell?

Do you think team members' presence in virtual meetings has changed in comparison to co-located meetings?

Do you think new, inappropriate behaviors have emerged? Can you identify some of the "new" behaviors, not as obvious during the live meetings?
 & 
 \begin{minipage}[t]{0.4\textwidth}
    \begin{itemize}
    \item Will give indirectly a more personal insight into the PL's perception of "inappropriate".
    \item Potential follow-up: Are these behaviors or habits repeated?
    \end{itemize}
  \end{minipage}
   \\
     Core & 
  	What was your first thought when seeing such a behavior?

What did you think about the student when seeing such a behavior? 

Was there a specific motivation or reason attached to such behavior? 

Was it necessary to take any measurements towards such behavior?

Did it have any effect on the way you approached the student, tasks were assigned, feedback was given?
	& 
 \begin{minipage}[t]{0.4\textwidth}
    \begin{itemize}
    \item Directly identify bias via this question. 
    \item This question aims to further explore the bias, and to retrieve information that could be used to build a persona.
    \item This question might help identify which behaviors were the most triggering
    \end{itemize}
  \end{minipage}
    \\
     Finisher 
     & What would you consider to be the ideal behavior and setting for a virtual meeting?
      & 
      \begin{minipage}[t]{0.4\textwidth}
    \begin{itemize}
    \item Summarizes the interview
    \item Finish on a positive note
    \end{itemize}
  \end{minipage}  \\
    \bottomrule
  \end{tabular}
\end{table}

After the first pilot interview, the results were analysed to identify strongest and weakest questions, and to restructure if necessary. We concluded that the biggest risk of the interview would be not reaching a satisfying level of model fit when identifying behaviours, but also reactions to such behaviours. Another result of the first test interview was to consider all sides of the attribution process, be this present or absent in a PL, and whether there are specific conditions leading to attribution. Discussions were made around the personality of the team members and how perceptions around their personality are generally formed. Identifying hindrance in a PLs ability to assess personality, and how fast perceptions are created in the remote setting could be potentially mapped to the proneness of forming attribution. Another aspect that could provide insights into the attribution process, is how trusting PLs are.

The finalized version of the interview guide focused primarily on identifying inappropriate behaviours and when possible, bias attached to these behaviours. Since it was not evident at start whether biases even exist, it was important to construct an interview in which the PLs could be presented with different kinds of behavioural categories, inquire if such behaviours were inappropriate and explore the effect of the behaviour in the perception created onto the team members. These eventually constituted the core questions of the interview. Guiding questions and follow-up questions aimed tat complementing the biases with more background, being this the effects of the remote setting in the team interaction, the relation of the PL with the team or the role personality of team members play in decision making. Such questions would also give the PLs the opportunity to express their own thoughts, impediments and worries, without intervening. Considering that semi-structured interviews can potentially become lengthy, we strived to keep a set of questions corresponding to 30 - 60 minutes per interview, to respectfully consider the participants' time schedules and constraints.  ~\autoref{tab:interviewFinalized} shows the finalized version of the interview.

\begin{longtable}{ p{0.2\textwidth} p{0.4\textwidth}   p{0.4\textwidth} }
  \caption[Finalized interview questions]{Finalized interview questions}
  \label{tab:interviewFinalized}\\
   \hline
    Question group  & Questions & Planned follow-up questions \\
    \hline
     Grand-tour  & 

    \begin{itemize}[leftmargin=*,topsep=0pt]
    \item How has the experience as a project leader in the remote setting been?  
    \item Do you think it is more difficult to establish a relationship between PL and student?
    \end{itemize}
 & 
 
     \begin{itemize}[leftmargin=*,topsep=0pt]
    \item What do you think has been the most difficult thing to handle in the interaction with the team regarding typical PL activities? 
    \item  What did you concretely change in your interaction with the team?
    \end{itemize}
 \\
     Core questions
     of inappropriate behaviour & 
     
      \begin{itemize}[leftmargin=*,topsep=0pt]
      \item Do you think the team is more uncomfortable or comfortable in the virtual setting? How would you tell?How has the experience as a project leader in the remote setting been?  
    \item  Are students less active in meetings as well?
    \item  Do you think the line between typical home activities and meetings have become more blurry? What are some of the things students do, they would not normally do in a normal meeting?
    \item  What specific behaviours have you observed, that make you think a student is distracted keeping in mind the virtual setting?
    \end{itemize}
  	
	& 
	 \begin{itemize}[leftmargin=*,topsep=0pt]
      \item Appearance-based: attire, unprofessional background
    \item  Communication-based: how the student communicates with the members, PLs and customers, having the camera off
    \item Distraction-based: notifications, another screen, another person in the room interacting with them, an animal
    \item Technology-based: poor internet connection, noisy microphone, not being "fast" with technology
    \end{itemize}
    \\
     Core questions to identify bias
     & 
      \begin{itemize}[leftmargin=*,topsep=0pt]
      \item Do you think x behaviour meant something about the student in the long term?
    \item  Were you ever mistaken about an attribution you made of a student?
    \item  How do you draw conclusions about a students personality? Do you even take personality into consideration?
    \end{itemize}

      & 
      \begin{itemize}[leftmargin=*,topsep=0pt]
      \item Were you ever wrong about a perception you created on the student? 
    \item Were there behaviors you thought were intentional? To annoy or irritate the PL or team. What did you think when a student exhibited x behaviour?
    \end{itemize}
    \\
    \bottomrule
\end{longtable}

\section{Generic Thematic Analysis Results}\label{Generic}

Coding qualitative research to find common themes and concepts is part of thematic analysis, which itself is part of qualitative data analysis. Thematic analysis extracts themes from text by analysing the word and sentence structure. There are several steps included in this process as described in \cite{Auerbach2003} and presented in Research Methodology \ref{ResearchMethodology}.

Although the focus of the thesis is the attribution error, interesting findings were identified on the experience of being a Project Leader (PL) in a pandemic, the casualty of remote meetings, and activeness of participants in such meetings. Therefore, the next three subsections refer to the above-mentioned themes. To provide a better overview of the coding process, tables representing the third step of the thematic coding process can be found in the Appendix for each of these themes. We will be referring to the codes from the interviews using the code names from~\autoref{tab:interviewDetails}.

\subsection{On the experience as a Project Leader}

Communicating and working remotely can be experienced differently from different project leaders. During the thematic analysis of the first questions posed at the PLs, several themes and patterns could be identified.

The lack of the social aspect dominated the answers, mentioning various situations or communication opportunities that are non-existent in the virtual setting. Small talk, especially when not related to the project’s topics, is missing and effecting the team dynamic. PLs mention that the time before and after the meeting, usually makes the team bond faster (I1). But even discussing project-related ideas is much more difficult without the on- and offboarding phase, as everyone becomes immediately unavailable. Such small talk is perceived also as an innovation driver, and many opportunities to bring ideas forward are left unexplored (I9). 

Lack of socialization not only grows the distance between team members, but it also grows the distance between team member – PLs (I9) and team member – project (I3)(I4). Project leaders find themselves investing more time in bringing the team closer through weekly ice breakers, non-technical check-ins or social media presence. Regarding the project, students cannot make the difference between the iPraktikum and a classic lecture anymore, considering the format now being the same, and are less invested in giving their best and even prioritizing this course over the other courses of the semester. This results in PLs continuously reminding the students of the gravity of the iPraktikum, which was unneeded in the pre-corona times.

Motivating the team was also one of the recurring topics in the interviews. Since it is now impossible to maintain eye contact with the team, is it also impossible to get the usual signals of a person having received a message, or information even reaching everyone. Therefore, how much attention team members are investing cannot be assessed anymore, and announcements, feedback and even appraisal might get lost (I4, I7, I8, I9, I10).

Body language is also something missing in the meetings. This "tool" previously helped PLs in assessing situations. Students' reactions are more difficult to see now (I4), and it's equally difficult to learn more about their personality (I5). Seeing team members as a whole person, would previously also contribute to building trust (I5). In in-person meetings it is easier to see what other people are feeling or thinking (I7), and therefore, react much sooner to situations that might escalate (I3).

\subsection{On the casualty of remote meetings}

Virtually attending team meetings and simply not having to be in a designated place, in which everyone has agreed to meet, intensifies the occurrence of distracting events, whether intentionally or unintentionally. 

After the interviewees were asked general questions on their experience, the interviews proceeded with questions that would stimulate thoughts on inappropriate behaviour, while still maintaining an abstract level. Such questions aimed at offering the interviewees the possibility to express their impressions openly, without being biased by the interviewer. The questions serving this goal were:

\begin{itemize}
\item Do you think the team is more uncomfortable or comfortable in the virtual setting?
\item Do you think the line between typical home activities and meetings have become more blurry?
\end{itemize}

Multiple themes emerged when answering these questions and 8 out of 10 project leaders agreed that being in a less professional environment, like a home, does have an impact.

One of the themes mentioned is the mindset. Although in a meeting, it is easier to think about non-work domestic topics, such as chores that need to be completed, or environmental stressors (I2). Being at home, can also put a person in a more relaxed mindset, making it difficult to switch from less demanding tasks to high focus and high productivity activities such as team meetings (I2, I4). This prevents the participants to fully be in the topic (I2). On the other hand, being at home makes switching to a more relaxed atmosphere easier, which can have benefits for students, as well as PLs (I3), but on the other hand, the flexibility of almost immediately logging off adds more distance and detachment to the project and team members too (I2, I4, I5). A discussion or meeting can even be left abruptly, especially when conflict occurs (I8), just as easily. 

It was mentioned that issues emerging from attending the meeting remotely are minimized in the cases in which the participants had a separate room or a designated office-like space (I4).

Another discomfort of remote meetings is latency issues due to the internet. It causes people to talk over each other (I6, I9) and cause distress for the participants, including the project leader. 

The remote setting makes it possible to attend a meeting, anywhere in the world, and there have been occasions in which the meetings were joined on the train (I2, I8,  I10), and in an extreme occasion while riding the bike, which points out to a tendency for treating such meeting more effortlessly. Being mobile, is usually accompanied with internet connection problems, difficulties understanding the other person due to the face mask (I10), and lack of participation in the said meeting.  

\subsection{On being active in remote meetings}

All participants of the interviews agreed that the remote setting postpones the moment of the ideal dynamic in the meetings. The first meetings are usually accompanied by slow sprint plannings (I6, I8), and participation of only few people in the discussions. PLs mention (I3, I7), that it is often necessary to directly ask questions at less active team members, to make sure their opinion is also taken into consideration. Especially at the beginning, whole teams might seem to be sleepy, which then requires the PL to intervene and wake the team up (I3).

As the weeks pass by, different perceptions are created. According to some PLs, the teams become more comfortable with each other and therefore, more comfortable to be active in the meetings (I3, I4, I8). At the beginning also more information gets lost (I8), and you are unsure if you are actually being heard or whether the others are actively listening (I3, I4, I9).

One pattern observed in a few interviews (I5, I7, I9) is the one of dominant, or over-active team members. Such individuals take over most of the discussions, are usually the first ones to state an opinion. Usually, such individuals may be naturally talkative and dominant (I7), or are already used to communicating remotely, potentially due to online collaborative video games (I1). 

Having such active team members, might overshadow the rest of the team members (I7). The ones that are less comfortable might turn off their camera or mute themselves (I7), making use or even abusing the virtual format of the meeting to be less participative (I1). In such a team dynamic, the more introverted people remain introverted (I2), by not stepping outside of their comfort zones (I2, I10). Therefore, online communication is perceived to not really develop communication skills (I2).

Project leaders also mentioned ways of dealing with such situations. In case the whole team seems distanced or distracted it might be beneficial to address that directly in the meeting, and motivating the team into efficient sprint planning (I3). When it comes to the hyperactive type of persona, a project leader might interrupt them by talking over them (I6) and therefore dominate the situation themselves. Another way to deal with extreme dynamics in a team, is to pair shyer people with more active ones, with the purpose of the shyer person to learn from the confidence of the more active person (I7). Usually, another way to address participation issues would be directly after the meeting, an occasion which is missing in the virtual setting. Here, the PLs distinguish by the urgency of the problem, and can decide to reach out to the student directly after the meeting (6), or talk about the issues in the 1 on 1 feedback sessions (I6, I7, I10). On the other hand, participation can be triggered in other, more entertaining ways such as organizing frequent ice breakers (I2, I3, I10) or checking in at the start of each meeting (I6). Also, in-person gatherings have been mentioned to have improved the atmosphere in the team and also participation in sprint planning noticeably. 


\section{Behaviour-based Thematic Analysis Results}\label{Behaviours}

Although the core questions were intentionally designed to identify inappropriate behaviours, 5 out of 10 participants would already mention concerns or dissatisfaction from various situations in the first questions of the interviews. These would then serve as a good basis for the following questions. Another noticeable point, is that some of the PLs were more comfortable with sharing their experience, and would talk openly about their dislikings, as for other PLs the familiarity would come later, if they had any experience to share. 

After conducting the test interview, it was decided that the interview would need more guidance in the section in which the PL would talk about their perception of inappropriate behaviour, therefore, a framework was introduced that would guide not only the analysis phase of the interviews, but also present interviewees with eventualities they maybe hadn't noticed before.

Upon transcribing the interviews, reading through their transcriptions, the coding and clustering phase followed, which motivated the identification of behavioural patterns throughout the interviews. Naturally, certain behaviours would be mentioned by a larger number of PLs, whereas others by one or two. Similarly, the gravity of the opinions shared would vary, as some interviewees would have stronger opinions, and others rely on phrases such as "could be", "for the others", to lessen the weight of their statements.

Since the number of behaviours mentioned are numerous, we are using the same categorization of the behaviours in the interviews questions to introduce them in a more structured manner. Examples were given though, mostly when the interviewee would ask for them. If a certain behaviour hasn't been addressed by certain PLs, it is so because they weren't faced with such a situation, or had a neutral opinion. 

\subsection{Zoom presence}

Being muted in remote meetings and therefore, not participating as actively, is perceived differently from different PLs. The person exhibiting this behaviour can be considered as as shy or introverted (I2), they might not be as experienced as others (I7), and therefore feel reluctant to share an opinion that might be wrong. It can also be a sign that the course is perceived as the typical lecture with a mandatory presence, in which the student "just" needs to listen in (I4). Not hearing too long from a person, it might also give the impression that they are doing something else instead (I9).

Not participating verbally can sometimes be accompanied by not participating visually either. Over Zoom, one has the opportunity to turn off the camera. Although having the camera turned off at all times wasn't mentioned to have happened that often, as the groups in the iPraktikum are smaller than a regular lecture, constantly switching the camera on and off, as a person is e.g. moving, or switching it off at a certain point of the meeting, still happens. Such a behaviour is considered to be confusing (I2), annoying (I2) and even disrespectful towards the others in the group (I9). In one interview it was referred to as an anti-pattern (I7), and another interviewee concluded that such a behaviour happens when the participant wants to hide what is actually occurring, and affects the way they view this person in a meeting (I3). 

Remote meetings give space to an increased level of mobility, which has been noticed from quite a few PLs. Although most of the participants seem to have separate rooms, or a stable area from which the meetings can be followed, there have been a few instances of joining meetings from outdoor locations, especially commuting from one place to another. Such occasions were often accompanied by a lack of participation and a distracted participant (I2, I4). Such behaviour can be considered unprofessional (I6), and shows that the team member isn't interested in getting the most out the iPraktikum (I4). In a team, it was obvious that this person was more involved in getting their side-tasks done, wherever they were, as opposed to being focused on the meeting (I2).

Participants can also be quite relaxed in the remote setting; sometimes too relaxed. Participants might appear to be sleepy (I3), which hurts the atmosphere in the meetings. They might also be slouching, expressing their boredom verbally and signalling a lack of attentiveness (I7). This is inappropriate, and they don't care about getting out of the comfort zone of their homes, according to I9. Being too comfortable also manifests itself through engaging in physical activity which is considered to be rude (I7) and unprofessional (I3, I7). A similar impression was given when a participant joined an important meeting from the comfort of their bed (I2).

Considering the backgrounds of the participants, it usually helps with learning more about the other person's life or hobbies (I5, I6, I9), but when disorganized, it might send the signal of being disorganized at work too (I5). This is especially impolite in formal meetings, as it is quite disrespectful towards the participants (I7).

Making your upper half-body visible is of importance as well, and if someone is showing only a small portion of themselves, it might say that they are doing something else instead which requires their attention (I9). Also the participants should at least look in the direction of the camera (I3).

\subsection{Communication}

Sometimes PLs are faced with situations in which participants rely on empty phrases to engage in the meetings, without actually contributing to the conversations or project. This might give the impression that this person is being too optimistic (I3), lazy (I3), uninterested in the project (I4). It makes it harder to asses this person's personality (I1). As they might be struggling with their performance, it might also not be in their culture to ask for support (I5).

Generally the communication to the customers has become a bit more relaxed, but once (I1) directly contacting them was considered to be inappropriate.

Although interrupting others is not uncommon in normal meetings, it is more obvious and problematic in remote meetings. In remote meetings, only one person can be heard at a certain time, which makes it easy to talk over the other person (I2). It is considered to be unprofessional and rude (I7) and it gives the impression they want to dominate the situation and make themselves the center of attention (I9). For one of PLs (I5), it was especially irritating when a participant "would not stop talking".

\subsection{Distractions}

Reasons for such a behaviour are numerous, but sometimes, "intruders" of remote meetings are visible in front of the camera too. A few PLs have noticed participants engaging in conversations with other people in their rooms. Such a behaviour is considered to be inappropriate (I3), unacceptable (I5), disrespectful (I7) and unprofessional (I9). It can be associated with not being interested in the meeting (I3) and it has a negative impact on the other participants as they might become distracted from such an interaction too (I7), and feel not being taken seriously (I9). One thing everyone could do, is to make an effort to minimize distractions, by letting other people know they shouldn't be interrupted (I5, I7). 

Sometimes distractions don't have to come in a human form. They can be right on the participants' screens. PLs notice when students are doing something else on their computers, and consider it to be inappropriate (I3), and rude (I4). It might also signal that the team member isn't eager to bring the project forward (I4).

Phones or Smart Watches are also big distractions, and it happens more often now that team members use their phone while being in meetings (I3, I5). This would not normally happen, according to I5. It gives the impression that the student isn't interested in the project, or in what the others are saying (I4). In a way, they choose to not pay attention (I4). Such a behaviour can be justified if the reason for such an interaction is of an urgent matter (I2). 

\subsection{Technology}

Sometimes using new technology is associated with acquiring new skills  and an opportunity to see more of a person, solely by the way they interact with it.

Not knowing how to connect to zoom or screenshare, is no excuse for a computer scientist (I2). On the other hand, viewing 20 open tabs on a persons browser leaves the impression that they probably opened the meeting a bit before the scheduled time (I3, I4) and are more interested in something else (I3). Sometimes people come unprepared, showing that the iPraktikum is not a priority to them (I4)

Problematic and rude is also when a person disagrees to use a certain technology, as it puts everyone in a difficult position and hinders the progress, according to I3. 

Less qualitative hardware might come in the way of meetings, and microphones seem to be a problem in a few teams. Having noisy microphones makes the other difficult to understand (I4, I5), and causes an unnecessary raise of voice in the meetings.

\subsection{Repeated behaviours}

Being late at meetings can occur in the virtual scenario too, but they seem to be less tolerable as there is no need to commute from one place to the other. It remains annoying (I2), rude (I3), and gives the impression that this person isn't aware of the scope of the iPraktikum (I1). Leaving early also isn't a good signal, as it shows that the person doesn't want to be there (I4).

Leaving unannounced is also something that remains unacceptable, and one of the very extreme cases of inappropriate behaviour, whether in virtual or in-person meetings (I3, I8). 

Being unprepared in meetings is inappropriate in any kinds of meetings, and often affects the way the others view you (I4). But, considering the comfortability of home and how there is more time available to prepare adequately, increases the criticism. According to I5, not adhering to deadlines is much less acceptable now.

It also got mentioned in three interviews (I5, I8, I10) that the tutor evaluations play a role in how Project Leaders perceive a student. This is an unchanged procedure from the on-site iPraktikum. 

\section{Conclusions}\label{Analysis}

After having looked at behaviours deemed unacceptable or inappropriate by the PLs of the iPraktikum21, it would be interesting to see whether certain groups have attained a bigger mentioning rate and which behaviours were more prone to the attribution bias than other behaviours.

Participation-avoiding behaviours would be the group of behaviours that would concern PLs the most. Joining from other places, camera and volume off, empty phrases would give the impression that the participants aren't really in the meetings - but rather pretend to be there. It is associated with disinterest and disrespect, among other things. 

Too-comfortable-at-home is also a pattern identified in the answers given, and, although a lot of activities such as drinking and eating were generally acceptable, others had more extreme responses such as attending from beds, slouching, engaging in physical activities or communicating with flatmates.

Meeting stoppers are also events which are not appreciated by the participants of the meeting, whether it's bad audio, difficulty connecting devices, internet connection disconnecting etc. These usually stop the flow of the meeting and often require spontaneous solutions.

Ego-centric behaviours such as speaking louder and often, interrupting others, contacting the customers directly, leaving the meeting unannounced are another obvious group. Although with less occurrences, these behaviours seem to have had a negative impact on the PLs.

From the thematic analysis of the interviews it can be seen that the attribution error exists, but it was difficult to track it directly with interviews, as different people would have different opinions and clustering responses together doesn't paint the whole picture. In interviews, the interviewer also needs to remain neutral in the conversations, and can't check cross-check all the mentioned biases with all interviewees. Some of the behaviours were also experienced by others and by others not, so reaching a common set of biases and behaviours was difficult. 

The attribution bias would be expressed in forms of being "rude", "disrespectful", "unprofessional", "lacking interest", to mention some of the most common expressions. These expressions lack situational attribution and refer to a person's attitude and personality, and served as a cue for the existence of dispositional attribution. Bias was present, especially in behaviours that were considered inappropriate before. But new ones also emerged, that were considered extremely inappropriate, even though only a few PLs had experienced them. 

After analysing the data, we considered it necessary to cross-check the most common behaviours against most common biases per behaviour. Through hypothetical situations, inspired by real instances, biases can be addressed more directly. Furthermore, the interviews shed light onto the experiences of PLs, and didn't include hypothetical situations, as it was important to gain a spectrum of behaviours first.

\section{Limitations and Validity of the Findings}\label{Limitations}

During a face-to-face interview, there is an opportunity to observe social and non-verbal cues of the interviewee. These cues may come in the form of voice, body language, gestures and intonation, and can supplement the interviewee’s verbal response and can give clues to the interviewer about the process of the interview. Similarly to the team meetings discussed in the interviews, the virtual nature of most interviews also made it difficult to properly read body language or other non-verbal cues. Nonetheless, the interviewees were comfortable, confident and eloquent in their responses. 

Additionally, although the goal of such interviews is to explore motives and patterns, the possibility to explore each one in detail was low, due to time and scoping issues. A compilation of interesting sub-topics is included (actually will be included) in Future Work.

Another impediment faced during the interviews was not receiving any problematic behaviors as answers. This case would result in meaningless answers to follow-up questions regarding the attribution process. In such situations, it is necessary to address the questions not linked to a specific behavior, in order to increase the chances of identifying bias. Luckily, only 3 out of the 10 interviews were particularly lower in inappropriate behaviors, therefore, not risking the purpose of the thesis.

Regarding the validation of the process and the data, we relied on the validity categories described in \cite{Runeson2012}.

The purpose of this phase of the thesis was to gain knowledge regarding situations and behaviours which are not only inappropriate, but moreover, trigger the phenomenon of attribution bias into a person. After the analysis of the data, we were able to identify a catalogue of 19 behaviours, and attribution biases attached to most of them. Furthermore, the interviewee statements required no further interpretation, that would threat the construct validity.

One factor influencing biases are the tutor assessments given to PLs at the beginning of the project. These did not play a role in identifying inappropriate behaviours in remote meetings, but they could play a role in forming biases, which then have an effect on how much attention a PL lays on the behaviours of a certain team member. These third factor could be one to be considered in future works in the topic.

Although considering a case study, the observations and theories from this phase could apply to similar, university-based projects, considering the age of the participants and the social-economical status. A special case of this case study is that is takes place during the pandemic of COVID-19, which introduced multiple changes in the lifestyle of the student population in Germany, regardless whether they were at TUM, or not.

Considering that the interview introduced a framework, which could be used for coding as well, we assume that the process of coding is valid and reproducible in a similar manner. Regarding clarity, the refinements made onto the second iteration of interview questions aimed at increasing the focus of the interview and addressing situations more directly, which we hope to have increased the reliability of the interviewing process.  